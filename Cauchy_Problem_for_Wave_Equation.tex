Дифференциальные уравнения с обыкновенными и, тем более, с частными производными имеют, вообще говоря, бесчисленное множество решений. Поэтому в том случае, когда физическая задача приводится к уравнению с частными производными, для однозначной характеристики процесса необходимо к уравнению присоединить некоторые условия.\\

В случае дифференциального уравнения 2-го порядка решение может быть определено начальными условиями, т. е. заданием значений функций и её первой производной при <<\textit{начальном}>> значении аргумента (задача Коши).
	\[\derp{u}{t}{2} = a^2 \derp{u}{x}{2}, \quad x \in R,\, t = 0\]
Начальные условия
	\begin{align*}
		&t=0 \quad u(0, x) = f(x)\\
		&t=0 \quad \derp{u}{t}{} (0, x) = g(x)\\
	\end{align*}
Если струна закреплена, то должны выполняться <<\textit{граничные условия}>>\\
	\begin{itemize} \setlength{\itemindent}{10pt}
		\item[Задача \textbf{первого} типа]
		$\begin{aligned}
			&u(t, 0)  = \varphi(t)\\
			&u(t, l) = \xi(t)\\
		\end{aligned}$ -- заданный режим
		\item[Задача \textbf{второго} типа]
		$\begin{aligned}
			&\derp{u}{x}{}(t, 0) = \varphi(t)\\
			&\derp{u}{x}{}(t, l) = \xi(t)\\
		\end{aligned}$ -- заданная сила
	
		\item[Задача \textbf{третьего} типа]
		$\begin{aligned}
			&\derp{u}{x}{}(t, 0) = \alpha(u + \varphi)\\
			&\derp{u}{x}{}(t, l) = \beta(u + \xi)\\
		\end{aligned}$ -- упругое закрепление
	\end{itemize}


