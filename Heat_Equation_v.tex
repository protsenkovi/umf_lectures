Пусть мы имеем тело  . Если его различные части имеют разные температуры, то возникают тепловые потоки от участков с более высокой температурой к участкам с более низкой температурой. Выберем систему координат  - температура тела в точке   в момент времени  .
На каком-нибудь элементе   на поверхности  , количество тепла, проходящее через него за промежуток времени   пропорционально произведению  и нормальной производной   от температуры  .
 											(1)
 - это функция   положительная б называется коэффициентом внутренней теплопроводности тела в точке  .
Будем считать, что   зависит только от точки   и не зависит от направления нормали  .Такое тело называется изотропным. 
Если через   обозначим тепловой поток, то есть количество тепла, проходящее через 1 единицу площади поверхности за единицу времени , то получим :
 .
Для вывода уравнения теплопроводности внутри тела   выделим объем  , ограниченный гладкой замкнутой поверхностью   и рассмотрим изменение количества тепла в объеме за промежуток  .Из уравнения (1):
   ,
где  -внутренняя нормаль к поверхности  .
Для изменения температуры элемента объема   на величину   за промежуток времени   нужно затратить тепло 
 ,
где  - плотность   в точке  ,
        - теплоемкость   в точке  .

 
Если внутри тела имеются источники или стоки, то обозначив   их плотность, можно определить количество тепла, выделяемого или поглощаемого в объеме   за промежуток времени  .
 .
Составим теперь уравнение теплового баланса для выделенного объема   , что дается равенством   .
  
 =  + .

Применяя теорему Остроградского к первому интегралу в правой части, получим :

  .
Комментарий:
Градиентом скалярной функции   векторного аргумента  , из евклидова пространства   называется, производная функции   по векторному аргументу  .
 
Дивергенцией вектора   называется -  .
Производная по направлению: .
 		(3)
В силу непрерывности подынтегральной функции и произвольного выбора   и   следует, что: 
 									(4)
 
  - Уравнение теплопроводности для однородного изотропного тела.
 								(5)
 ,  .
Если  , то получим однородное уравнение без источника:
  									(6) 
Для однородной пластины:
  										(7)
Для однородного стержня:
 .

Чтобы найти температуру тела в любой момент времени не достаточно одного уравнения, достаточно знать еще распределение температуры внутри тела в начальный момент времени (начальные условия) и тепловой режим на границе   (граничные условия)  в случае ограниченного тела.
В случае установившегося режима, когда температура не зависит от времени, то уравнение (5) переходит в уравнение Пуассона.
 
Если отсутствуют и тепловые источники, то получим уравнение Лапласа:

 .
Это уравнение Лапласа характеризует стационарный процесс , не зависящий от времени , для определения   не надо задавать начальное распределение температуры , а достаточно задать одно граничное условие , не зависимое от времени. Задача подобного типа называется задачей Дирихле.
Что касается граничных условий, они могут быть заданы различными способами:
1.В каждой точке поверхности   задается температура
  
 - точка поверхности  ;
 -известная функция при  .
2.На поверхности   задается тепловой  поток:
 
 
3.На поверхности твердого тела происходит теплообмен с окружающей средой, температура   которого известна:
 
Вывод:
Задача о распределении тепла в изотропном твердом теле ставится так:
Найти решение уравнения теплопроводности, удовлетворяющее начальному условию и одному из перечисленных трех граничных условий.
В одномерном случае, когда  , мы имеем стержень некоторой конкретной длины  .
Для вывода уравнения теплопроводности для простоты полагают, что стержень сделан из одного однородного теплопроводящего материала, боковая поверхность теплоизолированная ( тепло может распространиться только вдоль оси  ). Стержень тонкий , а значит что температура точек а каждом поперечном сечении постоянна.
