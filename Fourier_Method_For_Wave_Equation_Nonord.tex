Рассмотрим неоднородное уравнение колебаний
\begin{equation}
	\derp{u}{t}{2} = a^2 \derp{u}{x}{2} + f(x, t), \quad a^2 = \frac{k}{\rho}, \quad 0 < x < l
	\label{equ:FourierNonordinary1}
\end{equation}
с начальными условиями
\begin{equation}
	\left.
	\begin{aligned}
		u(x, 0) &= \varphi(x),\\
		u_t(x, 0) &= \psi(x),
	\end{aligned}
	\right\} \quad 0 \leqslant x \leqslant l
	\label{equ:FourierNonordinary2}
\end{equation}
и однородными граничными условиями
\begin{equation}
	\left.
	\begin{aligned}
		u(0, t) &= 0,\\
		u(l, t) &= 0,
	\end{aligned}
	\right\} \quad t > 0.
	\label{equ:FourierNonordinary3}
\end{equation}
Будем искать решение задачи в виде разложения в ряд Фурье по $x$
\begin{equation}
	u(x, t) = \sum\limits_{n = 1}^{\infty} u_n(t) \sin \frac{\pi n}{l}x,
	\label{equ:FourierNonordinary4}
\end{equation}
рассматривая при этом $t$ как параметр. Для нахождения $u(x, t)$ надо определить функцию $u_n(t)$. Представим функцию $f(x, y)$ и начальные условия в виде рядов Фурье:
\begin{equation}
	\left.
	\begin{aligned}
		f(x, t) &= \sum\limits_{n = 1}^{\infty} f_n (t) \sin \frac{\pi n}{l} x, &f_n(t) &= \frac{2}{l} \int\limits_0^l f(\xi, t) \sin \frac{\pi n}{l} \xi\, d \xi;\\
		\varphi(x) &= \sum\limits_{n = 1}^{\infty} \varphi_n \sin \frac{\pi n}{l}x, &\varphi_n &= \frac{2}{l} \int\limits_0^l \varphi(\xi) \sin \frac{\pi n}{l} \xi \, d\xi\\
		\psi(x) &= \sum\limits_{n = 1}^{\infty} \psi_n \sin \frac{\pi n}{l}x, &\psi_n &= \frac{2}{l} \int\limits_0^l \psi(\xi) \sin \frac{\pi n}{l} \xi \, d\xi
	\end{aligned}
	\right\} 
	\label{equ:FourierNonordinary5}
\end{equation}
Подставляя предполагаемую форму решения \eqref{equ:FourierNonordinary4} в исходное уравнение  \eqref{equ:FourierNonordinary1}
\[
	\sum\limits_{n = 1}^\infty \sin \frac{\pi n}{l} x \left\{- a^2 \left(\frac{\pi n}{l} \right)^2 u_n(t) - \ddot u_n (t) + f_n (t) \right\} = 0,
\]
видим, что оно будет удовлетворено, если все коэффициенты разложения равны нулю, т.е.
\begin{equation}
	\ddot u_n(t) + \left(\frac{\pi n}{l} \right)^2 a^2 u_n (t) = f_n (t).
	\label{equ:FourierNonordinary6}
\end{equation}

Для определения $u_n(t)$ мы получили обыкновенное дифференциальное уравнение с постоянными коэффициентами. Начальные условия дают:
\begin{align*}
	u(x, 0) = \varphi(x) = \sum\limits_{n = 1}^\infty u_n(0) \sin \frac{\pi n}{l} x = \sum\limits_{n = 1}^\infty \varphi_n \sin \frac{\pi n}{l} x,\\
	u_t(x, 0) = \psi(x) = \sum\limits_{n = 1}^\infty \dot u_n(0) \sin \frac{\pi n}{l} x = \sum\limits_{n = 1}^\infty \psi_n \sin \frac{\pi n}{l} x,
\end{align*}
откуда следует:
\begin{equation}
	\left.
	\begin{aligned}
		u_n(0) = \varphi_n,\\
		\dot u_n(0) = \psi_n.
	\end{aligned}
	\right\}
	\label{equ:FourierNonordinary7}
\end{equation}

Эти дополнительные условия полностью определяют решение уравнения \eqref{equ:FourierNonordinary6}. Функцию $u_n(t)$ можно представить в виде
\[
	u_n(t) = u_n^{(\mathrm{I})} (t) + u_n^{(\mathrm{II})} (t) 
\]
где
\begin{equation}
	u_n^{(\mathrm{I})} (t) = \frac{1}{\pi n a} \int\limits_0^t \sin \frac{\pi n}{l} a (t - \tau) \cdot f_n (\tau)\, d\tau
	\label{equ:FourierNonordinary8}
\end{equation}
есть решение неоднородного уравнения с нулевыми начальными условиями и 
\begin{equation}
	u_n^{(\mathrm{II})} (t) = \varphi_n \cos \frac{\pi n}{l} at + \frac{1}{\pi n a}\psi_n \sin \frac{\pi n}{l} a t
	\label{equ:FourierNonordinary9}
\end{equation}
--- решение однородного уравнения с заданными начальными условиями. Таким образом, искомое решение запишется в виде
\begin{align*}
	&u(x, t) = \sum\limits_{n = 1}^\infty \frac{1}{\pi n a} \int\limits_0^t \sin \frac{\pi n}{l} a (t - \tau) \sin \frac{\pi n}{l}x \cdot f_n(\tau)\, d\tau + \\
	&\phantom{u(x, t) =} + \int\limits_{n = 1}^\infty \left(\varphi_n \cos \frac{\pi n}{l} a t + \frac{1}{\pi n a} \psi_n \sin \frac{\pi n}{l} a t \right) \sin \frac{\pi n}{l}x.
\end{align*}

Вторая сумма представляет решение задачи о свободных колебаниях струны при заданных начальных условиях и была решена ранее. Обратимся к первой сумме, представляющей вынужденные колебания струны под действием внешней силы при нулевых начальных условиях. Пользуясь выражением \eqref{equ:FourierNonordinary5} для $f_n(t)$, находим:
\begin{align*}
	u_n^{(\mathrm{I})} (x, t) &= \int\limits_0^t \int\limits_0^l \left\{\frac{2}{l} \sum\limits_{n = 1}^\infty \frac{l}{\pi n a} \sin \frac{\pi n}{l} a(t - \tau) \sin \frac{\pi n}{l} x \sin \frac{\pi n}{l} \xi \right\} f(\xi, \tau)\, d\xi d\tau =\\
	&=\int\limits_0^t \int\limits_0^l G(x, \xi, t - \tau) f(\xi, \tau)\, d\xi d\tau,
\end{align*}
где 
\[
	G(x, \xi, t - \tau) = \frac{2}{\pi a} \sum\limits_{n = 1}^\infty \frac{1}{n} \sin \frac{\pi n}{l} a(t - \tau) \sin \frac{\pi n}{l} x \sin \frac{\pi n}{l} \xi.
\]
