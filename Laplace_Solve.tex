Большой интерес представляют решения уравнения Лапласа, обладающие сферической или цилиндрической симметрией, т.е. зависящие только от одно переменной $r$ или $\rho$.

Решение уравнения Лапласа $u = U(r)$, обладающие сферической симметрией, будет определяться из обыкновенного дифференциального уравнения
\[
	\der{}{r}{} \left(r^2 \der{U}{r}{} \right) = 0
\]
Интегрируя это уравнение, находим:
\[
	U = \frac{C_1}{r} + C_2,
\]
где $C_1$ и $C_2$ --- произвольные постоянные. Полагая, например, $C_1 = 1$, $C_2 = 0$, поулчаем функцию
\begin{equation}
	U_0 = \frac{1}{r},
	\label{equ:equFundLaplace1}
\end{equation}
которую называют \textit{фундаментальным решением уравнения Лапласа в пространстве}.

Аналогично, полагая
\[
	u = U(\rho)
\]
и пользуясь уравнением Лапласа в цилиндрических координатах
\[
	\Delta_{\rho, \varphi, z} u = \frac{1}{\rho} \derp{}{\rho}{} \left(\rho \derp{u}{\rho}{} \right) + \frac{1}{\rho^2} \derp{u}{\varphi}{2} + \derp{u}{z}{2} = 0,
\]
найдём решение, обладающее цилиндрической или круговой симметрией (в случае двух независимых переменных), в виде
\[
	U(\rho) = C_1 \ln \rho + C_2.
\]
Выбирая $C_1 = -1$ и $C_2 = 0$, будем иметь:
\[
	U_0 = \ln \frac{1}{\rho}
\]
Функцию $U_0(\rho)$ часто называют \textit{фундаментальным решением уравнения Лапласа на плоскости} (для двух независимых переменных).

%Функция $U_0 = \frac{1}{r}$ удовлетворяет уравнению $\Delta u = 0$ всюду, кроме точки $r = 0$, где она обращается в бесконечность. С точностью до множителя пропорциональности она совпадает с полем точечного заряда $e$, помещённого в начале координат
