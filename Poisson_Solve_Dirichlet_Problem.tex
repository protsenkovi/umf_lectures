Теперь будем считать, что A стремится к центру шара.

$u = \frac{1}{4 \pi} \int_0^{2 \pi}$\\

Утверждение. 
Непрерывно дифференциируемая гармоническая функция $u$ может принимать максимальное и минимальное значение только на границе области.
Доказательство от противного.

Ограничиваем эту точку сферой радиуса $\epsilon$. На поверхности сферы найдём максимальное значение. По нашему предположению $u^* > u'$. Тогда $u^* = \frac{1}{4 \pi R^2} \int\limits_S u'\; dS$. Заменяем $u$ на самое большое значение. Если хотя бы в одной точке 
По теореме о среднем пришли к противоречию.

Согласно 
этой теореме решение задачи Дирихле -- единственно.

Предположим, что задача имеет два решения.
Опираясь на 1 теорему, максимальные и минимальные значения принимает на границах, следовательно во всей области.