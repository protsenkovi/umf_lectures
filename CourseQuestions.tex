
\begin{enumerate}
\setlength\parsep{0ex} 
\setstretch{0.0}
\setlength\itemsep{0ex} \small
\item Уравнение малых поперечных колебаний струны. c.~\pageref{que:1}\\
\item Уравнение теплопроводности. c.~\pageref{que:2}\\
\item Дифференциальные уравнения с двумя независимыми переменными. с.~\pageref{que:3}\\
\item Оператор Лапласа в декартовых, полярных, цилиндрических и сферических координатах. с.~\pageref{que:4}\\
\item Определения линейных, квазилинейных, однородных и неоднородных уравнений. с.~\pageref{que:5}\\
\item Классификация дифференциальных уравнений в частных производных второго порядка. с.~\pageref{que:6}\\
\item Приведение уравнений с двумя независимыми переменными к каноническому виду. с.~\pageref{que:7}\\
\item Постановка задачи Коши для волнового уравнения. Метод Даламбера для бесконечной струны. с.~\pageref{que:8}\\
\item Начальные и граничные условия для полубесконечной струны. Метод Даламбера для полубесконечной струны. с.~\pageref{que:9}\\
\item Метод разделения переменных или метод Фурье. Задача о колебаниях струны с закрепленными концами. с.~\pageref{que:10}\\
\item Метод разделения переменных или метод Фурье. Задача для гиперболического неоднородного уравнения с начальными и однородными граничными условиями. Функция влияния. с.~\pageref{que:11}\\
\item Редукция общей краевой задачи для волнового уравнения. с.~\pageref{que:12}\\
\item Определение сопряженных дифференциальных операторов. с.~\pageref{que:13}\\
\item Колебания прямоугольной пластины с начальными и однородными граничными условиями. с.~\pageref{que:14}\\
\item Задача Штурма-Лиувилля. Ортогональность собственных функций. с.~\pageref{que:15}\\
\item Метод Фурье в задаче Коши для уравнения параболического типа. с.~\pageref{que:16}\\
\item Фундаментальное решение уравнения теплопроводности. с.~\pageref{que:17}\\
\item Редукция общей задачи теплопроводности для конечного стержня. с.~\pageref{que:18}\\
\item Вывод формул Грина. с.~\pageref{que:19}\\
\item Фундаментальное решение уравнения Лапласа. с.~\pageref{que:20}\\
\item Принцип максимума уравнения Лапласа. с.~\pageref{que:21}\\
\item Единственность и устойчивость краевой задачи Дирихле для уравнения Лапласа.  с.~\pageref{que:22}\\
\item Задача Неймана для уравнения Лапласа. с.~\pageref{que:23}\\
\item Первая краевая задача для круга. с.~\pageref{que:24}\\
\item Метод функции Грина для задачи Дирихле в трехмерном случае. с.~\pageref{que:25}\\
\item Метод функции Грина для задачи Дирихле в двумерном случае. с.~\pageref{que:26}\\
\item Уравнение Лежандра. Полиномы Лежандра. с.~\pageref{que:27}\\
\item Ортогональность функций Лежандра. с.~\pageref{que:28}\\
\item Присоединенные функции Лежандра. с.~\pageref{que:29}\\
\item Ортогональность присоединенных функций Лежандра. с.~\pageref{que:30}\\
\item Уравнение Бесселя. Функции Бесселя. с.~\pageref{que:31}\\
\item Ортогональность Функций Бесселя. с.~\pageref{que:32}\\
\item Гамма функция. Основные свойства Гамма функции. с.~\pageref{que:33}\\
\item Разделение переменных в трехмерном уравнении Лапласа в сферических координатах. с.~\pageref{que:34}\\
\item Вариационная формулировка краевых задач. с.~\pageref{que:35}\\
\item Вариационный метод Ритца. с.~\pageref{que:36}\\
\end{enumerate}


