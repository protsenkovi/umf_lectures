Возникает необходимость в способах, позволяющих достаточно просто рассчитывать и сложные системы.

Один из возможных путей состоит в применении простых приближенных формул (например, формулы Рэлея). В этом случае задают форму колебаний системы, сводя её таким образом к системе с одной степенью свободы. При удачной аппроксимации получают достаточно точное значение низшей собственной частоты системы, однако другие её динамические характеристики остаются нераскрытыми.

Cхематизация реальной системы, как имеющей несколько степеней свободы, достигается в методе Рэлея-Ритца, при использовании которого форма колебаний системы задаётся в виде выражения, включающего несколько параметров.

Другим приёмом, позволяющим свести реальную систему к системе с конечным числом степеней свободы, является метод прямой дискредитации. Чем больше число элементов, на которые разбита система при использовании этого метода, тем ближе расчётная схема к исходной системе. Вместе с тем, если элементы выбраны однотипными, то даже при большом их числе оказывается возможным реализовать расчёт колебаний, используя матричные методы с применением ЭВМ. Примерами таких методов являются метод начальных параметров в форме матриц перехода и метод прогонки.

При динамических расчётах конструкций сложной конфигурации также широко используется метод конечных элементов.

В том случае, когда сложную колебательную систему можно разделить на несколько подсистем, динамические характеристики которых определяются сравнительно просто, полезными являются методы динамических податливостей и жёсткостей. Эти методы представляют собой обобщение на динамические задачи метода сил и метода перемещений строительной механики.

В методе последовательных приближений задача об определении собственных частот и форм колебаний сводится к многократному расчёту деформаций системы под действием известной статической нагрузки.