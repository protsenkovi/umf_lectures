%Будем рассматривать однородное линейное уравнение второго порядка
%\[
%	Ly \equiv  a_2(x) y'' + a_1(x)y' + a_0(x)y = 0.
%\]
 %Его можно записать по-другому: 	
 %\begin{equation}
%	Ly \equiv \der{}{x}{} \left[p(x) \derp{y}{x}{} \right] - q(x) = 0.
%	\label{equ:equShturmLeuv1}
 %\end{equation}


%Однородное уравнение $Ly = 0$ и неоднородное $Ly = f$, как известно, имеют бесконечное множество решений. На практике часто бывает нужно из множества решений выделить только одно. Для этого задают некоторые дополнительные условия. Если это начальные условия $y(x_0) = y_0, y'(x_0) = y_1$, то получают задачу Коши. Если задают дополнительные условия на концах некоторого отрезка, то получают задачу, которая называется краевой задачей. Условия, которые задаются на концах отрезка, называются краевыми условиями. Краевые условия иногда именуют также граничными условиями и тогда говорят о граничной задаче.
%Мы будем задавать линейные краевые условия вида
% \begin{equation}
%	\left\{
%	\begin{aligned}
%		&l_1 y \equiv \alpha_1 y(x) + \alpha_2 y'(a) = A,\\
%		&l_2 y \equiv \beta_1 y(b) + \beta_2 y'(b) = B
%	\end{aligned}
%	\right.
%	\label{equ:equShturmLeuv2}
 %\end{equation}



%Наряду с уравнением \eqref{equ:equShturmLeuv1} 
\subsubsection{Общий вид}
Рассмотрим уравнение\footnote{\href{http://vicaref.mgsu.ru/ODE/lec4.html}{Источник}}	
 \begin{equation}
	L_\lambda y \equiv \der{}{x}{} \left[p(x) \derp{y}{x}{} \right] - [\lambda r(x) - q(x)] y = 0,
	\label{equ:equShturmLeuv3}
 \end{equation}
содержащее некоторый числовой параметр $\lambda$. Здесь функции $p(x), q(x), r(x)$ действительные, а число $\lambda$ может быть, вообще говоря, и комплексным. Краевая задача \eqref{equ:equShturmLeuv3}, \eqref{equ:equShturmLeuv2} при $A = B = 0$ является однородной. Поэтому при любых $\lambda$ она имеет тривиальное решение. Нас будут интересовать такие значения $\lambda$, при которых эта задача обладает не только тривиальными решениями.\\
Будем задавать линейные краевые условия вида
 \begin{equation}
	\left\{
	\begin{aligned}
		&l_1 y \equiv \alpha_1 y(x) + \alpha_2 y'(a) = A,\\
		&l_2 y \equiv \beta_1 y(b) + \beta_2 y'(b) = B
	\end{aligned}
	\right.
	\label{equ:equShturmLeuv2}
 \end{equation}
где $\alpha_1, \alpha_2, \beta_1, \beta_2, A, B$ - заданные числа, причем по крайней мере одно из чисел $\alpha_1, \alpha_2$, и одно из чисел $\beta_1, \beta_2$, отличны от нуля. Если в \eqref{equ:equShturmLeuv2} хотя бы одно из чисел $A$ и $В$ не равно нулю, то краевые условия называют неоднородными. Если $A = B = 0$, то условия \eqref{equ:equShturmLeuv2} называются однородными.\\ %Краевая задача называется однородной, если рассматривается однородное уравнение \eqref{equ:equShturmLeuv1} $Ly = 0$ и однородные краевые условия \eqref{equ:equShturmLeuv2}. Решением краевой задачи называется такое решение дифференциального уравнения, которое удовлетворяет заданным краевым условиям. Заметим сразу, что однородная краевая задача всегда имеет решение $y \equiv 0$ (тривиальное решение).


\textbf{Задача Штурма-Лиувилля.} \textit{Найти те значения параметра $\lambda$, при которых уравнение \eqref{equ:equShturmLeuv3} имеет нетривиальное решение, удовлетворяюшее однородным краевым условиям \eqref{equ:equShturmLeuv2}. В дальнейшем будем ее записывать в виде}
\[
	L_\lambda y = 0,  l_1y = 0,  l_2y = 0.
 \]

         Те значения параметра $\lambda$, при которых задача Штурма-Лиувилля имеет ненулевое решение, называются \textit{собственными значениями} (\textit{собственными числами}) задачи, а сами эти решения - \textit{собственными функциями}. Задачу Штурма-Лиувилля называют также задачей на собственные значения. В силу однородности уравнения и краевых условий собственные функции задачи Штурма-Лиувилля определены с точностью до постоянного множителя. Это означает, что если $y(x)$ -собственная функция при некотором значении $\lambda$, то произведение $Cy(x)$, где $C$ - произвольная постоянная, также является собственной функцией при том же значении параметра $\lambda$. В связи с этим часто в качестве собственной функции рассматривают нормированную функцию $y(x)$, у которой $||y(x)|| = 1$. Такая собственная функция определена, по существу, однозначно (с точностью до знака $\pm$). Далее рассмотрим наиболее простой случай задачи Штурма-Лиувилля, когда уравнение имеет вид   
 \begin{equation}
	y'' + \lambda  y = 0.
	\label{equ:equShturmLeuv4}
 \end{equation}


Из множества краевых условий вида \eqref{equ:equShturmLeuv2} ограничимся тремя частными случаями:
\begin{enumerate} \setlength\itemsep{0ex}
	\item краевые условия первого рода   \[y(a) = y(b) = 0\]
	\item краевые условия второго рода  \[y'(a) = y'(b) = 0\]
	\item краевые условия третьего рода 
	\[
		\left\{
			\begin{aligned}
				&y'(a) = \sigma_1 y(a)\\
				&y'(b) = \sigma_2 y(b), \quad \sigma_1 > 0, \quad  \sigma_2 > 0.
			\end{aligned}
		\right.
	\]
\end{enumerate}

%Общая задача Штурма-Лиувилля будет обладать свойствами, очень похожими на свойства в этих простых случаях, если на коэффициенты уравнения \eqref{equ:equShturmLeuv3} наложить дополнительные условия: $p(x), q(x), f(x)$ -- непрерывные функции, причем $p(x)$ имеет, кроме того, непрерывную производную на $[a, b], p(x) > 0, q(x) \geqslant 0$.
%свойства
\subsubsection{Ортогональность собственных функций}
	\begin{theo}
		Система собственных функций регулярной задачи Штурма-Лиувилля ортогональна на $[a, b]$ с весом $r(k)\footnote{\href{http://www.our-lectures.ru/exams/matfizika/201-zadacha-shtyrma-liyvilla-ortogonaknost-i-razlojenie.html}{Источник}}$:
\[
	\int\limits_a^b r(x) y_m y_k\, dx = \begin{cases} 0, & m \neq k\\ ||y_n||^2, & m = k\end{cases}
\]
	\end{theo}
\begin{qproof}
\begin{align*}
	 \left[p(x) X_m'(x) \right]' + \left[\lambda_m r(x) - q(x) \right]X_m (x) &= 0 \quad \Big| \cdot X_k\\
	 \left[p(x) X_k'(x) \right]' + \left[\lambda_k r(x) - q(x) \right]X_k (x) &= 0 \quad \Big| \cdot X_m
\end{align*}
\begin{align*}
	X_k(x) \left[p(x) X'_m(x)  \right]' - X_m(x) \left[p(x) X'_k(x)  \right]' + (\lambda_m - \lambda_k) r(x) X_k(x) X_m(x) &= 0\\
	\derp{}{x}{} \left[p(x) \left(X_m' X_k - X_m X_k' \right) \right] + (\lambda_m - \lambda_k) r(x) X_k(x) X_m(x) &= 0\\
	\underbrace{ p(x)  \left(X_m' X_k - X_m X_k' \right) \Big|_a^b }_{\displaystyle= 0 \quad \mbox{для ур-й 1, 2, 3 рода}} + (\lambda_m - \lambda_k) \int\limits_a^b r(x) X_k(x) X_m(x) &= 0
\end{align*}
Следовательно
\[
	(\lambda_m - \lambda_k) \int\limits_a^b r(x) X_k(x) X_m(x) = 0
\]
\[
		\int\limits_a^b r(x) y_m y_k\, dx = \begin{cases} 0, & m \neq k\\ ||y_n||^2, & m = k\end{cases}
\]
\end{qproof}
