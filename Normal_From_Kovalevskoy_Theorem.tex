\setcounter{equation}{0}

\begin{defin}[Аналитическая функция]
	Функция в точке $(t, x_1^0, x_2^0, \ldots , x_n^0)$ аналитическая, если в окрестности этой точки её можно разложить в равномерно сходящийся ряд:\\
		\[f(t, x_1, x_2, \ldots, x_n) = \sum\limits_{k_0k_1 \ldots k_n = 0}^{\infty} C_{k_0k_1 \ldots k_n} \prod\limits_{k=1}^{n} (x_i - x_i^0)^{k_i} (t - t_0)^{k_0}\]
		где
		\[C_{k_0 k_1  \ldots k_n} = \frac{1}{k_0!k_1! \ldots k_n!} \frac{\partial^{k_0 + k_1 + \cdots + k_n}f}{\partial t^{k_0}\partial x_1^{k_1} \ldots \partial x_n^{k_n}}\]
	\end{defin}


	%\begin{qproof}\\
	%	В силу сходимости ряда мы можем найти $M$, такое, что $C_{k_0 k_1 ... k_n} a_0^{k_0} a_1^{k_1} ... a_n^{k_n} < M$ тогда\\
	%	$C_{k_0 k_1 \ldots k_n} < \frac{M}{ a_0^{k_0} a_1^{k_1} \ldots a_n^{k_n}}$, тогда ряд\\
	%	$F = \sum \frac{M}{ a_0^{k_0} a_1^{k_1} \ldots a_n^{k_n}} (t - t_0)^{k_0} (x_1 - x_1^0)^{k_1} (x_2 - x_2^0)^{k_2} \ldots (x_n- x_n^0)^{k_n}$\\
	%\end{qproof}
	\begin{defin}[Система уравнений типа Ковалевской\footnote{Известный русский математик Cофья Васильевна Ковалевская }] 

	Система уравнений типа Ковалевской имеет вид:
%\[
%	\derp{u}{t}{n_i} =  F \left(t, x_1, x_2, \ldots,  \frac{\partial^{k_0 + k_1 + \cdots + k_n}f}{\partial t^{k_0}\partial x_1^{k_1} \ldots \partial x_n^{k_n}}\right) = 0
%\]
\begin{equation}
	\derp{f_i}{t}{n_i} = F \left(t, x, f, \ldots, \frac{\partial^{k_0 + k_1 + \cdots + k_n}f}{\partial t^{k_0}\partial x_1^{k_1} \ldots \partial x_n^{k_n}} \right)
	\label{equ:equNormal}
\end{equation}
где
	\begin{align*}
		k_0 + k_1 + \dots + k_n &\leqslant n_i\\
		k_0 &< n_i
	\end{align*}
\end{defin}

	Обычно $t$ играет роль времени, а $x_1, x_2, \ldots, x_n$ -- пространственные координаты. При $t = t_0$ задаются значения неизвестных функций $f_i$ и производных порядка $n_i - 1$.

\begin{equation}
	\left. \derp{u_i}{t}{k} \right| = \varphi_i^{(k)} (x), \quad k = 0, 1, \ldots, n_i - 1
	\label{equ:equCauchyProblem}
\end{equation}
$\varphi_i^{(k)}$ заданы в $G \subset R^n$

	При этих условиях уравнение \eqref{equ:equNormal} называется \textit{уравнением нормального вида}, а задача решения уравнения \eqref{equ:equNormal}, удовлетворяющего начальным условиям \eqref{equ:equCauchyProblem}, называется \textit{задачей Коши}.\\


	Систему уравнений Ковалевской можно свести к системе уравнений первого порядка:\\
	
\[
\begin{cases}
	\displaystyle\derp{u}{t}{2} = a_1 \derp{u}{x}{2} + a_2 \derps{u}{x}{y} + a_3 \derp{u}{y}{2} + a_4 \derps{u}{t}{x} + a_5 \derps{u}{t}{y} + b_1 \derp{u}{t}{} + b_2 \derp{u}{x}{} + b_3 \derp{u}{y}{} + Cu + f\\
	\displaystyle u(t_0) = \varphi (x_1, x_2, \dots, x_n)\\
	\displaystyle\derp{u}{t}{}|_{t = t_0} = \xi (x_1, x_2, \dots, x_n)
\end{cases}
\]

	Замены: $\derp{u}{t}{} = u_0 \quad \derp{u}{x}{} = u_1 \quad \derp{u}{y}{} = u_3$\\
		\[\derp{u_0}{t}{} = a_1 \derp{u_1}{x}{} +  a_2 \derp{u_2}{x}{} +  a_3 \derp{u_2}{y}{} +  a_4 \derp{u_0}{x}{} +  a_5 \derp{u_0}{y}{}\]
		\[t = t_0 \quad u_0 = \xi (x, y)\]
		\[\derp{u_1}{t}{} = \derp{u_0}{x}{} \quad u_1 = \derp{\varphi}{x}{} (x,y)\]
		\[\derp{u_2}{t}{} = \derp{u_0}{y}{} \quad u_2 = \derp{\varphi}{y}{}(x,y)\]


\begin{theo}[Теорема Ковалевской\footnote{ Теорема была представлена в 1874 году в Гёттингенский университет (вместе с двумя другими работами) под названием “Zur Theorie der partiellen Differential-Gleichungen” в качестве докторской диссертации и опубликованная в 1875 году в “Journal fur die reine und angewandte Mathematik” (Berlin. Bd. 80. S. 1–32), она явилась первым значительным результатом в общей теории уравнений с частными производными. До тех пор глубоко изучались главным образом уравнения математической физики, то есть отдельные примеры уравнений с частными производными, возникающие в конкретных физических задачах, как, например, уравнение теплопроводности, описывающее распределение тепла в нагретом теле, уравнение колебания струны или мембраны, уравнение распространения звуковых колебаний, уравнение Лапласа, описывающее многие физические процессы электропроводимости, гидродинамики, стационарной теплопроводности и др. Можно считать, что работа Ковалевской положила начало развитию общей теории уравнений с частными производными.}]
	Если в некоторой точке $(t_0, x_1^0, \dots ,x_n^0)$ все функции, входящие в систему -- аналитические, то в малой окрестности точки существует решение задачи Коши и при этом -- единственное.\\
\end{theo}
\begin{qproof}\\
Без доказательства
%	Без нарушения общности будем считать $t_0 = 0$. Также $\xi(\theta) = \varphi (\theta) = 0$\\
%		\[v_0 u_0 - \xi\]
%		\[v_1 = u_1 - \varphi_x'\]
%		\[v_2 = u_2 - \varphi_y'\]
%		\[\derp{v_1}{t}{} + \derps{\varphi}{t}{x} = \derp{v_0}{x}{} + \derp{\varphi}{x}{}\]
%		\[u_i (t, x, y) = u_i (\theta) + \derp{u_i}{t}{}|_{t=0} t + \derp{u_0}{x}{} |_{x=0} x + \derp{u_i}{y}{} |_{y=0} y\]
%		+ $\dots$\\
%	Для определения единственности надо показать, что коэффициенты в ряду определяются однозначно. Первый коэффициент одназначно определяется из начальных условий.\\
%	Так как мы оперируем с аналитическими функциями, мы можем найти любые производные\\
%		\[\derp{u_i}{t}{} = \sum\limits_{k=1}^{2} \sum\limits_{j = 0} \derp{u_j}{x_k}{} a_{jk} + \sum\limits_{j = 0} b_j u_j + f_i\]
%		\[t=0 \quad u_i = \varphi_i (x,y) = \varphi_i(x_1,x_2)\]
%	Аналитические функции можно разложить в равномерно сходящиеся степенные ряды.\\
%		\[\xi_m = \sum\limits_{k_0, k_1, k_2 = 0}^{\infty} C_{k_0, k_1, k_2}^{(m)} t x_1^{k_1} x_2^{k_2}\]
%	так как 
%		\[
%			C_{k_0, k_1, k_2} |t_0|^{k_0} |x_{10}|^{k_1} |x_{20}|^{k_2} \leqslant M^{(m)}
%		\]
%		\[ 
%			W^{*(m)} = \sum\frac{M^{(m)}}{|x_{10}|^{k_1} |x_{20}|^{k_2}} |t_0|^{k_0} t^{k_0} x_1^{k_1} x_2^{k_2}
%		\]
%		\[
%			W^{*(m)}_1 = \sum\frac{M^{(m)}}{ (1 - \abs{\frac{t}{t_a}})^{k_0} (1 - \abs{\frac{x_1}{x_{10}}})^{k_1} (1 - \abs{\frac{x_2}{x_{20}}})^{k_2}}
%		\]
%		\begin{multline*}
%			\frac{1}{(1 - \abs{\frac{t}{t_0}})^{k_0}} = \left(1 + \left( - \abs{\frac{t}{t_0}} \right)\right)^{-k_0} = 1 + k_0  \abs{\frac{t}{t_0}} + (- k_0)(-k_0 - 1) \left(-  \abs{\frac{t}{t_0}} \right)^2 \frac{1}{2} + {}\\ {}+ (- k_0) (- k_0 - 1) (-k_0 - 2) \left(-  \abs{\frac{t}{t_0}} \right)^3 \frac{1}{3!}
%		\end{multline*}
%		\[
%			W_1^{*(m)} = \sum \frac{k! M^{(m)}}{k_1! k_2! k_3!} \frac{t^{k_0} x_1^{k_1} x_2^{k_2}}{|t_0|^{k_0} |x_{10}|^{k_1} |x_{20}|^{k_2}}
%		\]
%		\[
%			W_1^{*(m)} > W^{*(m)} \quad k! \geqslant k_0! k_1! k_2!
%		\]
%		\[
%			W^{*(m)} = \frac{M^{(m)}}{1 - \frac{t + x_1 + x_2}{a}}, \quad a = min(\abs{t_0}, \abs{x_{10}}, \abs{x_{20}})
%		\]
%		\[
%			W^{*(m)} = M^{(m)} (1 - \frac{\frac{t}{a} + x_1 + x_2}{a}) = \sum \frac{M^{(m)} k!}{k_0! k_1! k_2!} \frac{t^{k_0} x_1^{k_1} x_2^{k_2}}{\alpha^{k_0} a^k}
%		\]
%		\[
%			\xi_m \leqslant W^{*(m)} = \frac{M^{(m)}}{1 - \cfrac{\frac{t}{a} + x_1 + x_2}{a}}
%		\]
%		\[
%			M = \underset{m}{max} M^{(m)},
%		\]
%		тогда
%		\[
%			\forall_m \xi_m \leqslant (1 - \frac{\frac{t}{a} + x_1 + x_2}{a})
%		\]
%	Искомая функция $u_i \leqslant u (1 -  \frac{\frac{t}{a} + x_1 + x_2}{a})^{-1}$\\
%		\[
%			\derp{u_i}{t}{} \leqslant W \left( 6 \derp{u}{x}{} + 3 u + 1 \right)
%		\]
%	Заменой
%		\[
%			\frac{du}{\alpha dz} = W (-6 \der{u}{z}{} + 3 u + 1)
%		\]
%		\[
%			\frac{u}{3u + 1} = - \frac{\alpha dz W}{1 - 6 \alpha W}
%		\]
%		\[
%			\frac{1}{3} \ln (3 u + 1) = \alpha \int\limits_{0}^{z} \frac{W dz}{ 6 \alpha W - 1} = B (z)
%		\]
%		\[
%			u(z) = \frac{1}{3} (e^{3B(z)} - 1)
%		\]
%		\[
%			B(z) = \alpha \int W(z) (1 - \alpha 6 W(z))^{-1} dz
%		\]
%		\[
%			B(z) \mbox{принимает малые значения}
%		\]
%		\[
%			u(z) = \frac{1}{3} (e^{3B(z)} - 1) = \frac{2}{3} (3 B(z) + \frac{9 B^2(z)}{2} + \dots)
%		\]
\end{qproof}
