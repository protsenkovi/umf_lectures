\begin{figure}
	\centering	
	\includegraphics{7.jpg}
	\caption{}
\end{figure}
	при $z = 0 \quad u = f(x, y)$\\
	Для данной задачи мы должны построить функцию Грина.
	$G = W - \frac{1}{r_{AP}} = \frac{B}{r_{A*P} - \frac{1}{r_{AP}}}$\\
	$G|_{z = 0} = \left(\frac{B}{r_{A*P} - \frac{1}{r_{AP}}} \right)_{z = 0} = 0$\\
	$r_{AP} \sqrt{(x - x_0)^2 + (y - y_0)^2 + (z - z_0)^2}$\\
	$r_{A*P} \sqrt{(x - x_0)^2 + (y - y_0)^2 + (z + z_0)^2}$\\
	$z = 0 \to r_{AP} = r_{A*P}$\\
	нормаль в сторону противоположную $z - \Rightarrow \derp{G}{n}{} = - \derp{G}{z}{}$\\
	$G = \frac{1}{r_{A*P}} - \frac{1}{r_{AP}}; - \derp{G}{z}{} = \frac{1}{r_{A*P}^2} \derp{r_{A*P}}{z}{} - \frac{1}{r_{AP}^2} \derp{r_{A*P}}{z}{} = \frac{(z + z_0)}{r_{A*P}^3} - \frac{(z - z_0)}{r_{AP}^3} |_{z = 0} = \frac{2 z_0}{r_{AP}^3}$\\
	Перейдём к формуле Грина (3).\\
	$4 \pi u(x_0, y_0, z_0) = \iint\limits_{S} u \derp{G}{n}{} dS = - \int\limits_{- \infty}^{+ \infty} dx \int\limits_{- \infty}^{+ \infty} dy f(x, y) \derp{G}{z}{} = - 2 z_0 \int\limits_{- \infty}^{+ \infty} dx \int\limits_{- \infty}^{+ \infty} \frac{f(x,y)}{\left[(x - x_0)^2 + (y - y_0)^2 +z_0^2 \right]^\frac{3}{2}} dy =
	 [(x - x_0)^2 + (y - y_0)^2 +z_0^2 = x^2 + y^2 + x_0^2 + y_0^2 + z_0^2 - 2x x_0 - 2 y y_0 = \rho^2 + \rho_0^2 - 2 \rho \rho_0 \cos \theta \cos \theta_0 - 2 \rho \rho_0 \sin \theta \sin \theta_0] = - 2 z_0 \int\limits_{0}^{2 \pi} d\theta \int\limits_{0}^{\infty} \frac{f(\rho, \theta) \rho d\rho}{[\rho^2 + \rho_0^2 - 2 \rho \rho_0 \cos \theta \cos \theta_0 - 2 \rho \rho_0 \sin \theta \sin \theta_0]^\frac{3}{2}}$\\
	$\iint\limits_S \left( u \derp{G}{n}{} - G \derp{u}{n}{}\right)dS + \iiint\limits_{\omega} = \begin{cases} 
						4 \pi u(x_0, y_0, z_0) & if A inner \\
						2 \pi u(x_0, y_0, z_0) & if A border\\
						0 & if A \notin \bar \omega
					\end{cases}$\\

Для пространственной задачи фундаментальным решением является $\frac{1}{r}$\\
В плоском случае фундаментальным решением является $\ln \frac{1}{r}$\\
$\Delta u = 0 \quad \derp{u}{x}{2} + \derp{u}{y}{2} = 0$\\
$r = \sqrt{(x - x_0)^2 + (y - y_0)^2}$\\
$u = \ln \frac{1}{r} = - \ln r$\\
Найдём производную\\
$\derp{u}{x}{} = - \frac{1}{r} \derp{r}{x}{} = - \frac{1}{r^2}$\\
$\derp{u}{x}{2} = - \frac{1}{r^2} + \frac{2 (x - x_0)^2}{r^4}$\\
$\derp{u}{yx}{2} = - \frac{1}{r^2} + \frac{2 (y - y_0)^2}{r^4}$\\
$\derp{u}{x}{2} = \derp{u}{y}{2} = - \frac{2}{r^2} + \frac{2 [(x - x_0)^2 + (y - y_0)^2]}{r^4} = - \frac{2}{r^2} + \frac{2}{r^2} = 0$\\

Функцию Грина будем строить в  виде:\\
$G = W - \ln \frac{1}{r_{AP}}$\\
В случае, если решаем задачу Дирихле, то для функции $G$:
$\Delta G = 0$\\
$G|_S = 0$\\

Формула (2) будет иметь вид:\\
$\oint\limits_{GR} \left(u \derp{v}{n}{} - v \derp{u}{n}{}\right) d \gamma = \iint\limits_S \left(u \Delta v - v \Delta u \right) dS$\\
$\oint\limits_{GR} \left(u \derp{v}{n}{} - v \derp{u}{n}{}\right) d \gamma + \oint\limits_{GR_1} \left(u \derp{v}{n_1}{} - v \derp{u}{n_1}{}\right) d \gamma = 0$\\
\includegraphics{8.jpg}
Если интеграл берётся по окружности $\varepsilon$, то $\derp{}{n_1}{} = - \derp{}{r}{} \quad d \gamma = \varepsilon d \theta$\\
$\oint\limits_{GR_1} \left(u \derp{v}{n_1}{} - G \derp{u}{n_1}{}\right) d \gamma = \int\limits_{0}^{2 \pi} \left(-u \derp{v}{r}{} - G \derp{u}{r}{}\right) \varepsilon d \theta = \int\limits_{0}^{2 \pi} \left(- u \derp{W}{r}{} + W \derp{u}{r}{} + u \derp{}{r}{} (\ln \frac{1}{r}) - \ln \frac{1}{r} \derp{u}{r}{}\right) \varepsilon d \theta$\\
$(- \ln r)' = - \frac{1}{r}$\\
Перейдём к пределу\\
$\lim\limits_{\varepsilon \to 0} \varepsilon  \int\limits_{0}^{2 \pi} \left(-u \derp{W}{r}{} + W \derp{W}{r}{} \right) d \theta - \lim\limits_{\varepsilon \to 0} \varepsilon \int\limits_{0}^{2 \pi} [u \frac{1}{\varepsilon} + \derp{u}{r}{} \ln \varepsilon] d \theta$\\
Непрерывные фукнции
так как интегрируем по очень маленькой области u можно заменить u с точкой.