\textbf{Задача Дирихле}:
	Дана область $\Omega$, ограниченная поверхностью $\Sigma$. Требуется найти функцию $u(M)$, которая 
\begin{enumerate}
	\item определена и непрерывна в замкнутой области $\bar \Omega = \Omega + \Sigma$;\\
	\item удовлетворяет в открытой области $\Omega$ уравнению Лапласа: $\Delta u = 0$;
	\item принимает на поверхности $\Sigma$ заданное значение: $u\Big|_\Sigma = f(p), \quad p \in \Sigma$. 
\end{enumerate}
\begin{theo}[Единственность решения задачи Дирихле]
	Первая внутренняя краевая задача для уравнения Лапласа имеет единственное решение.	
\end{theo}

\begin{qproof} (от противного)\\
Допустим, что существуют два решения $u_1(M)$ и $u_2(M)$ первой краевой задачи.\\

Рассмотрим их разность:   $u(M) = u_1(M) - u_2(M)$. \\

Функция $v(M)$ удовлетворяет требованиям 1 и 2 решения задачи Дирихле:   $v$ - определена и непрерывна в $\bar \Omega$; $\Delta v = 0$ в $\Omega$. На границе  получаем: $v\Big|_\Sigma = 0$.\\ 

Функция $v$ в замкнутой области достигает своего максимального значения. \\

Если предположить, что $v > 0$ хотя бы в одной внутренней точке, то получим, что максимальное значение $v$ достигается внутри $\Omega$, что противоречит принципу максимума (так как $v\Big|_\Sigma = 0$) для гармонических функций. \\

Аналогичные рассуждения можно провести для случая, когда $v < 0$ в $\Omega$. Получим противоречие с принципом минимума для гармонических функций. 
\end{qproof}

\begin{theo}[Устойчивость решения задачи Дирихле]
	Задача Дирихле устойчива относительно малый возмущений на границах области.
\end{theo}
\begin{qproof}
	$u_1$ и $u_2$ --- гармонические функции.
	\begin{align*}
		\Delta u_1 = 0 &\Delta u _2 = 0\\
		u_1 \big|_S = f & u_2 \big|_S = f + \varepsilon
	\end{align*}
	\[
		\varepsilon \ll 1, \abs{u_1 - u_2} < \varepsilon
	\]
\end{qproof}

