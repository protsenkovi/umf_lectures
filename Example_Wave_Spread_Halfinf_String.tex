Рассмотрим задачу о распространении волн на полуограниченной прямой $x \geqslant 0$. Эта задача имеет особенно важное значение при изучении процессов отражения волн от конца и ставится следующим образом:\\

\textit{найти решение уравнения колебаний}
\[
	\derp{u}{t}{2} = a^2\, \derp{u}{x}{2} 
\]
\textit{при $0 < x < \infty$,\quad $t > 0$,}\\
\textit{удовлетворяющее граничному условию}
\[
	u(0, t) = \mu(t) \:(\mbox{или}\: \derp{u}{x}{}(0, t) = \nu(t)) \quad t \geqslant 0
\]
\textit{и начальным условиям}
\begin{alignat*}{1}
	u(x, 0) &= f(x)\\
	\derp{u}{t}{}(x, 0) &= g(x)
\end{alignat*}
$0 \leqslant x < \infty$.\\

Отметим две леммы о свойствах решений уравнений колебаний, определённых на бесконечной прямой.
\begin{enumerate}
	\item \textit{Если начальные данные в задаче о распространении колебаний на неограниченной прямой} (\textit{задача} \eqref{equ:equInfStringCauchy})  \textit{являются нечётными функциями относительно некоторой точки $x_0$, то соответствующее решение в этой точке $x_0$ равно нулю.}
	\item \textit{Если начальные данные в задаче о распространении колебаний на неограниченной прямой}  (\textit{задача} \eqref{equ:equInfStringCauchy}) \textit{являются чётными функциями относительно некоторой точки $x_0$, то производная по $x$ соответствующего решения в этой точке равна нулю.}
\end{enumerate}


Примем $x_0$ за начало координат, $x_0 = 0$. В этом случае условия нечётности начальных данных ($f(x)$ и $g(x)$ - нечётные) запишутся в виде 
\[
	f(x) = - f(-x); \quad g(x) = - g(- x).
\]
Функция $u(x, t)$, определяемая формулой \eqref{equ:equDalamber}, при $x = 0$ и $t > 0$ равна
\[
	u(0, t) = \frac{f(at) +f(-at)}{2} + \frac{1}{2a} \int\limits_{-at}^{at} g(\xi) d\xi = 0
\]
так как первое слагаемое равно нулю в силу нечётности $f(x)$, а второе равно нулю, поскольку интеграл от нечётной функции в пределах, симметричных относительно начала координат, всегда равен нулю.\\

Аналогично для второй леммы. Условия чётности начальных данных имеют вид
\[
	f(x) = f(-x); \quad g(x) = g(- x).
\]
Заметим, что производная чётной функции является функцией нечётной
\[
	\varphi'(x) = - \varphi '(- x)
\]
Из формулы \eqref{equ:equDalamber} следует:
\[
	u_x(0, t) = \frac{f'(at) + f'(-at)}{2}  + \frac{1}{2a} [g(at) - g(-at)] = 0, \quad t > 0,
\]
так как первое слагаемое равно нулю в силу нечётности $f'(x)$, а второе - в силу чётности $g(x)$.\\
Рассмотрим граничное условие
\[
	u(0, t) = 0, \quad t > 0
\]
Функцию $f(x)$ можно продолжить  нечётным образом:
\[
	f_1(x) = sgn(x) \cdot f(\abs{x})
\]
Аналогично для 
\[
    g_1(x) = sgn (x) \cdot g(\abs{x})
\]

\begin{enumerate}
	\item $x - at > 0; \quad$ -- решение записывается в обычном виде\\ (в области $t < \frac{x}{a}$ влияние граничных условий не сказывается и выражение для $u(x, t)$ совпадает с решением \eqref{equ:equDalamber} для бесконечной прямой.)\\
	\item $x - at < 0; \quad  x > 0,\quad  t > \frac{x}{a}$
\end{enumerate}
Получим решение уравнения колебаний
\begin{multline*}
	u(x, t) = \frac{f_1(x + at) + f_1(x - at)}{2} + \frac{1}{2a} \int\limits_0^{x +at} g_1(\xi)\, d \xi + \frac{1}{2 a} \int\limits_{x - at}^0 g_1 (\xi )\, d\xi =\\= [-z = \xi \quad -dz = d\xi ] 
	= \frac{f_1(x + at) + f_1(x - at)}{2} +  \frac{1}{2a} \int\limits_0^{x +at} g_1(\xi)\, d \xi + \frac{1}{2 a} \int\limits_0^{at -x} g_1 (z )\, dz
\end{multline*}
или
\[
	u(x, t) = \frac{f_1(x + at) + f_1(x - at)}{2} + \frac{1}{2a} \int\limits_{x-at}^{x+at} g_1(\xi)\, d\xi
\]
Сформулируем метод продолжений:\\

\textit{Для решения задачи на полубесконечной прямой с граничным условием $u(0, t) = 0$ начальные данные надо продолжить на всю прямую нечётно.}

\textit{Для решения задачи на полубесконечной прямой с граничным условием $u_x(0, t) = 0$ начальные данные надо продолжить на всю прямую чётно.}\\


