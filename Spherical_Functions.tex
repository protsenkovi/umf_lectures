Сферические функции проще всего могут быть выведены при решении уравнения Лапласа для шаровой области методом разделения переменных (метод Фурье). \\
Мы рассмотрим вывод на примере решения задачи теплопроводности.

Будем искать решение уравнения в переменных $r, \theta, \varphi$
\[
    \derp{u}{t}{} = a^2 \left(\derp{u}{r}{2} + \frac{2}{r} \derp{u}{r}{} + \frac{1}{r^2 \sin \theta} \derp{}{\theta}{} \left(\sin \theta \derp{u}{\theta}{}\right) + \frac{1}{r^2 \sin^2 \theta} \derp{u}{\varphi}{2}\right)
\]

\begin{align*}
      &0 \leqslant r \leqslant R\\
	&0 \leqslant \theta \leqslant \pi\\
	&0 \leqslant \varphi \leqslant 2 \pi
\end{align*}



Связь с декартовой системой координат
\begin{align*}
      &x = r \sin \theta \cos \varphi\\
	&y = r \sin \theta \sin \varphi\\
	&z = r \cos \theta
\end{align*}
Положим 
\[
    u(t, r, \theta, \varphi) = T(t) v (r, \theta, \varphi)
\]
Решим задачу теплопроводности
\[
    T' v = a^2 T \Delta_{r, \theta \varphi} v \Big| \frac{1}{a^2 v T}
\]

\[
    \frac{T'}{a^2 T} = \frac{\Delta_{r, \theta, \varphi} v}{v} = - k^2
\]

\[
    \frac{T'}{a^2 T} = - k^2
\]

\[
    \Delta_{r, \theta  \varphi} v = - k^2 v
\]
$v(r, \theta, \varphi)$ будем искать в виде
\[
    v(r, \theta, \varphi) = R(r) Y(\theta, \varphi)
\]
Составим уравнение
\[
   \left. R'' Y + \frac{2}{r} R' + \left[\frac{1}{r^2 \sin v} \der{}{\theta}{} \left( \sin \theta \derp{Y}{\theta}{}\right)  + \frac{1}{r^2 \sin^2 \theta} \derp{Y}{\varphi}{2}\right]R(r) = - k^2 R Y \quad \right| \frac{r^2}{R Y}
\]

\[
    \frac{r^2 R'' + 2 r R'}{R} + k^2 r^2 = - \frac{\left[ \frac{1}{\sin \theta} \derp{}{\theta}{} \left( \sin \theta \derp{Y}{\theta}{} \right) + \frac{1}{\sin^2 \theta} \derp{Y}{\varphi}{2}\right]}{Y} = \lambda^2
\]


\begin{equation}
    R'' + \frac{2}{r} R' + \left(k^2 - \frac{\lambda^2}{r^2}\right) R = 0
	\label{equ:equBess}
\end{equation}
Уравнение \eqref{equ:equBess} сводится к уравнению Бесселя.

\[
    \derp{Y}{\theta}{2} + \ctg \theta \derp{Y}{\theta}{} + \frac{1}{\sin^2 \theta} \derp{Y}{\varphi}{2} + \lambda^2 Y = 0
\]
Решение задачи $Y(\theta, \varphi)$ также ищем методом разделения переменных
\[
    Y_\lambda = F(\varphi) P(\theta)
\]

\[
    \left. P'' F + \cos \theta P' F + \frac{1}{\sin^2 \theta} F'' P + \lambda^2 F P = 0\quad \right| \frac{\sin^2 \theta}{P F}
\]
Разделим переменные
\[
    \frac{\sin^2 \left[P'' + \cos \theta P' \right]}{P} + \lambda^2 \sin^2 \theta = - \frac{F''}{F} 
\]
В итоге получим
\[
    - \frac{F''}{F} = m^2 
\]
$m^2$ -- константа\\

$F$ удовлетворяет уравнению
\[ 
     F'' + m^2 F = 0
\]
%и условию переодичности
%\[
%    F(\varphi + 2 \pi) = F(\varphi)
%\]
Решается в виде
\[ 
     F = A_m \sin m \varphi + B_m \cos m \varphi
\]

\[
    \left. \sin^2 \theta P'' + \sin \theta \cos \theta (\lambda^2 \sin^2 \theta - m^2) P = 0 \quad \right|  Y_\lambda = F(\varphi) P(\theta)
\]
Замена
\[
    \cos \theta = z \quad \sin^2 \theta = (1 - z^2)
\]
Найдём производные
\begin{align*}
    &\der{P}{\theta}{} = \der{P}{z}{} \der{z}{\theta}{} = - \sin \theta \derp{P}{z}{}\\
    &\der{P}{\theta}{2} = - \cos \theta \der{P}{z}{} + \sin^2 \theta \der{P}{z}{2}
\end{align*}

\[
    P'' + \cos P' + \left(\lambda - \frac{m^2}{\sin^2 v} \right) P = 0
\]

\[
    (1 - z^2)P''_{zz} - \cos \theta \der{P}{z}{} + \cos \theta \left(- \sin \theta\right) \der{P}{z}{} + \left(\lambda - \frac{m^2}{1 - z^2}\right) P
\]

В итоге получили
\[
     (1 - z^2) \der{P}{z}{2} - 2 z \der{p}{z}{} + \left(\lambda - \frac{m^2}{1 - z^2} \right) P = 0
\]
