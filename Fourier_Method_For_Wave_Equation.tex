\setcounter{equation}{0}
Метод разделения переменных или метод Фурье, является одним из наиболее распространённых методов решения уравнений с частными производными. Изложение этого метода проведём для задачи о колебаниях струны, закреплённой на концах.\\
\begin{equation}
	\derp{u}{t}{2} = a^2\, \derp{u}{x}{2}
	\label{equ:equCauchyWaveFourier}
\end{equation}
Начальные условия: 
\begin{align}
	u(x, 0) = f(x)\\
	u_t(x, 0) = g (x)
	\label{equ:equCauchyWaveFourierN}
\end{align}
Граничные условия:
\begin{align}
	u(0, t) = 0, \quad u(l, t) = 0
	\label{equ:equCauchyWaveFourierGr}
\end{align}
Уравнение \eqref{equ:equCauchyWaveFourier} линейно и однородно, поэтому сумма частных решений также является решением этого уравнения. Имея достаточно большое количество частных решений, можно попытаться при помощи суммирования их с некоторыми коэффициентами найти искомое решение. \\

Поставим основную вспомогательную задачу:\\
\textit{найти решение уравнения \eqref{equ:equCauchyWaveFourier} не тождественное нулю, удовлетворяющее однородным граничным условиям}
\begin{align*}
		u(0, t) = 0\\
		u(l, t) = 0
\end{align*}
\textit{и представимое в виде произведения }
\begin{equation}
	u(x,t)= X(x) T(t)
	\label{equ:equWaveFourierSolveView}
\end{equation}
\textit{где $X(x)$ -- функция только переменного $x$, $T(t)$ -- функция только переменного $t$.}\\

Подставляя предполагаемую форму решения \eqref{equ:equWaveFourierSolveView} в уравнение \eqref{equ:equCauchyWaveFourier}, получим:
\[
	X''T = \frac{1}{a^2} T''X
\]
после деления на $XT$
\[
	\frac{X''}{X} = \frac{1}{a^2} \frac{T''}{T} = - k^2\
\]
Из этого соотношения получаем обыкновенные дифференциальные уравнения для определения функций $X(x)$ и $T(x)$
\begin{alignat}{2} \label{equ:equCauchy1}
	X''(x) + k X(x) &= 0, \quad &X(x) &\not\equiv 0\\
	T''(t) + a^2 k T(t) &= 0, \quad &T(t) &\not\equiv 0
	\label{equ:equCauchy2}
\end{alignat}

Приходим к задаче о собственных значениях (задаче \textit{Штурма---Лиувилля}).
\[
	\begin{cases}
		X'' + kX = 0\\
		X(0) = X(l) = 0
	\end{cases}
\]
При $k < 0$ и $k = 0$ задача не имеет нетривиальных решений. Это можно проверить найдя решения этих двух случаев, общий вид которых $X(x) = C_1 e^{\sqrt{-k}x} + C_2 e^{-\sqrt{-k}x}$ и $X(x) = C_1 x + C_2$ соответственно.
Рассмотрим случай при $k > 0$.

Общее решение ищется в виде
\[
	X_k = C_{1k}^* \cos kx + C_{2k}^* \sin k x
\]
Граничные уловия дают:
\[
	X(0) = 0 \Rightarrow C_1^* = 0
\]
\[
	X(l) = 0 \Rightarrow C_{2k}^*\sin k l = 0
\]
\[
	k l=\pi n \Rightarrow k_n=\pi\frac{n}{l}
\]
$k_n$ -- cобственные числа.

Этим собственным числам соответствую собственные функции ($n \in N$)
\[
	X_n= C_{2n}^* \sin \frac{\pi n}{l} x
\]
Этим же значениям $k_n$ соответствуют решения уравнения \eqref{equ:equCauchy2}
\[
	T_n = A_k\cos \frac{a\pi k}{l} t + B_k \sin\frac{a\pi k}{l} t,
\] 
где $A_k$ и $B_k$ -- произвольные постоянные.
\begin{align*}
	&A_n C_{2n}^* =C_{1n}\\
	&B_n C_{2n}^* = C_{2n}
\end{align*}
Функции
\[
	u_n (x, t) = \left(C_{1n} \cos \frac{a \pi n}{l} t + C_{2n} \sin \frac{a \pi n}{l} t \right) \sin\frac{\pi n}{l} x
\]
являются частными решениями уравнения \eqref{equ:equCauchyWaveFourier}, удовлетворяющими граничным условиям \eqref{equ:equCauchyWaveFourierGr}. Эти решения могут удовлетворить начальным условиям нашей исходной задачи только для частных случаев начальных функций $f(x)$ и $g(x)$. \\

В силу линейности и однородности уравнения \eqref{equ:equCauchyWaveFourier} сумма частных решений
\begin{equation}
	u(x, t) = \sum\limits_{n = 1}^{\infty} u_n (x, t) = \sum\limits_{n = 1}^{\infty} \left(C_{1n} \cos \frac{a \pi n}{l} t + C_{2n} \sin \frac{a \pi n}{l} t \right) \sin\frac{\pi n}{l} x
	\label{equ:equUniSolveFourWave}
\end{equation}
также удовлетворяет этому уравнению и граничным условиям. 

Начальные условия позволяют определить $C_{1n}$ и $C_{2n}$. Потребуем, чтобы функция \eqref{equ:equUniSolveFourWave} удовлетворяла условиям \eqref{equ:equCauchyWaveFourierN}:
\begin{equation}
	\left.
	\begin{aligned}
		&u(x, 0) = f(x) = \sum\limits_{n = 1}^{\infty}  u_n(x, 0) = \sum\limits_{n = 1}^{\infty} C_{1n} \sin \frac{\pi n}{l} \, x,\\
		&u_t(x, 0) = g(x) = \sum\limits_{n = 1}^{\infty}  \derp{u_n}{t}{}(x, 0) = \sum\limits_{n = 1}^{\infty} \frac{\pi n}{l} a C_{2n} \sin \frac{\pi n}{l} \, x.
	\end{aligned}
	\right\}
	\label{equ:equwaveFourier1}
\end{equation}

Из теории рядов Фурье известно, что произвольная кусочно-непрерывная и кусочно-дифференциируемая функция $\varphi(x)$, заданная в промежутке $0 \leqslant x \leqslant l$, разлагается в ряд Фурье
\[
	\varphi(x) = \sum\limits_{n = 1}^\infty b_n \sin \frac{\pi n}{l} x,
\]
где 
\[
	b_n = \frac{2}{l} \int\limits_0^l f(\xi) \sin \frac{\pi n}{l}\xi \, d\xi
\]

Найдём коэффициенты $C_{1n}$
\[
	 C_{1n} = \frac{2}{l} \int\limits_0^l f(x) \sin \frac{n \pi}{l} x\, dx
\]
Коэффициенты $C_{2n}$ находят из 2-го условия $\derp{u}{t}{} |_{t = 0} = g(x)$\\
\[
	g(x) = \left. \sum\limits_{n = 1}^{\infty}\left(C_{2n}\cos \frac{a \pi n}{l} t + C_{1n} \sin \frac{a \pi n}{l} t \right) \frac{a \pi n}{l} \sin \frac{\pi n}{l} x \right|_{t = 0}
\]
\[
	g(x) = \sum\limits_{n = 1}^{\infty} C_{2n} \frac{a \pi n}{l} \sin \frac{\pi n}{l} x
\]
\[
	\frac{2}{l} \int\limits_0^l g(x) \sin \frac{\pi n}{l} x dx = \frac{a \pi n}{l} C_{2n} 
\]
\[
	C_{2n} = \frac{2}{a \pi n} \int\limits_0^l g(x) \sin \frac{\pi n}{l} x\, dx
\]
В итоге
\begin{align*}
	&C_{1n} = \frac{2}{l} \int\limits_0^l f(x) \sin \frac{n \pi}{l} x\, dx\\
	&C_{2n} = \frac{2}{a \pi n} \int\limits_0^l g(x) \sin \frac{\pi n}{l} x\, dx
\end{align*}\\
Проверим, совпадает ли данное решение с решением Д'Аламбера.

\[	
	u(x, t) = \frac{f(x + at) + f(x - at)}{2} + \frac{1}{2a} \int\limits_{x - at}^{x + at} g(\xi)\, d \xi
\]
\[
	f(x) = \sum\limits_{n = 1}^{\infty} C_{1n} \sin \frac{\pi n}{l} x \quad g(x) = \sum\limits_{n = 1}^{\infty} C_{2n} \frac{a \pi n}{l} \sin \frac{\pi n}{l} x\
\]
\begin{multline*}
	u(x,t) = \frac{1}{2} \left[ \sum\limits_{n = 1}^{\infty} C_{1n} \sin \frac{\pi n}{l} (x - at) + \sum\limits_{n = 1}^{\infty} C_{1n} \sin \frac{\pi n}{l} (x + at)\right] + \frac{1}{2a} \int\limits_{x - at}^{x + at} \sum\limits_{n = 1}^{\infty} C_{2n} \frac{a \pi n}{l} \sin \frac{\pi n}{l} \xi \, d \xi =\\
	= \sum\limits_{n = 1}^{\infty} C_{1n} \sin \frac{\pi n}{l} x \cos at \frac{\pi n}{l} +\frac{\pi}{a 2 l} \sum\limits_{n = 1}^{\infty} \int\limits_{x - at}^{x + at} C_{2n} n \sin\frac{\pi n}{l}\xi \, d \xi =\\
	=\left. \sum\limits_{n = 1}^{\infty}C_{1n} \sin \frac{\pi n}{l} x \cos \frac{a \pi n}{l} t - \frac{a \pi}{a 2 l} \sum\limits_{n = 1}^{\infty} C_{2n} \frac{l}{\pi} \cos \frac{\pi n}{l} \right|_{x - at}^{x + at} =\\
	= \sum\limits_{n = 1}^{\infty} \left( C_{1n}\sin \frac{\pi n}{l} x \cos\frac{a \pi n}{l} t + C_{2n} \sin \frac{\pi n}{l} x \cdot \sin \frac{\pi n a}{l} t\right) =\\
	= \sum\limits_{n = 1}^{\infty} \left(C_{1n} \cos \frac{\pi n a}{l} t +  C_{2n} \sin \frac{\pi n a}{l} t \right) \sin \frac{\pi n}{l} x
\end{multline*}
Таким образом убедились в правильности решения.