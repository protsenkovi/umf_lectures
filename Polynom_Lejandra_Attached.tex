\subsubsection{Присоединённые функции} \setcounter{equation}{0}
В задачах \textit{Штурма-Лиувилля} в многомерном случае уравнение Лежандра принимает вид
\begin{equation}
    \der{}{z}{} \left[ (1 - z^2) \der{y}{z}{} \right] +\left[n(n + 1) - \frac{m^2}{1 - z^2}\right] y = 0
		\label{equ:LejandrAtt1}
\end{equation}
где $m$ --- константа.

Будем искать решение $y$ в виде
\[
    y = (1 - z)^{\frac{m}{2}} v (z)
\]
Найдём первую производную
\[
    \der{y}{z}{} = \frac{m}{2} (1 - z^2)^{\frac{m - 2}{2}} (- 2 z) v + (1 - z^2)^{\frac{m}{2}} \der{v}{z}{}
\]
Умножим на $(1 - z^2)$
\[
    (1 + z^2) \der{y}{z}{} = - m z(1 - z^2)^{\frac{m}{2}} v + (1 - z^2)^{\frac{m + 2}{2}} \der{v}{z}{}
\]
\begin{multline*}
    \der{}{z}{} \left[ (1 -z^2) \der{y}{z}{} \right] = - m (1 - z^2)^{\frac{m}{2}} v + m \frac{m}{2} z (2 z) (1 - z^2)^{\frac{m - 2}{2}} v - mz(1 - z^2)^{\frac{m}{2}} \der{v}{}{} -\\ {}- \frac{m + 2}{2} (2 z) (1 - z^2)^{\frac{m}{2}} \der{v}{z}{} + (1 - z^2)^{\frac{m + 2}{2}} \der{v}{z}{2}
\end{multline*}
Подставим в формулу  \eqref{equ:LejandrAtt1}
\begin{multline*} 
    (1 - z^2)^{\frac{m + 2}{2}} \der{v}{z}{2} + \der{v}{z}{} \left[ - m z (1 - z^2)^\frac{m}{2} + (m + 2)z (1 - z^2)^\frac{m}{2} \right] +{}
	\\+ v \left[- m (1 - z^2)^\frac{m}{2} + m^2 z^2 (1 - z^2)^\frac{m - 2}{2} + n(n + 1)(1 - z^2)^2\frac{m}{2} - \frac{m^2}{1 - z^2} (1 - z^2)^\frac{m}{2} \right] = 0,
\end{multline*}
сократим на $(1 - z^2)^\frac{m}{2}$ 
\begin{equation}
    (1 -z^2) \der{v}{z}{2} - 2 z (m + 1) \der{v}{z}{} + \left[n(n + 1) - m(m + 1) \right] v = 0
	\label{equ:LejandrAtt2}
\end{equation}

Возьмём
\[
    (1 - z^2)\der{y^*}{z}{2} - 2 z \der{y^*}{z}{} + n(n + 1) y^* = 0, \qquad y^* = P_n(z)
\]
Возьмём первую производную\\
первое слагаемое
\[
    \der{}{z}{m} \left[(1 - z^2)\der{y^*}{z}{2} \right] = \der{y^*}{z}{m + 2} + m \der{y^*}{z}{m+1} (- 2 z) + \frac{m(m - 1)}{2} \der{y^*}{z}{m} (- 2) = 0
\]
второе слагаемое
\[
    \der{}{z}{m} \left[z \der{y^*}{z}{} \right] = \der{y^*}{z}{m + 1} z + m \der{y^*}{z}{m} 
\]

\begin{equation*}
	\der{y^*}{z}{m + 1} (- 2 z) + \frac{m(m + 1)}{2} \der{y^*}{z}{m} 
\end{equation*}

\begin{equation}
    (1 - z^2) \der{y^*}{z}{m + 2} - 2 z (m + 1) \der{y^*}{z}{m + 1} + \left[- m^2 + m - 2 m + n(n + 1) \right] \der{y^*}{z}{m} = 0
	\label{equ:LejandrAtt3}
\end{equation}

Если $v = \der{y^*}{z}{m}$, то \eqref{equ:LejandrAtt2} $=$ \eqref{equ:LejandrAtt3}
\[ 
	v = \der{y^*}{z}{m} = \der{}{z}{m} \left( P_n(m) \right)
\]
Подставив в уравнение \eqref{equ:LejandrAtt1}, получим решение
\begin{equation}
    y = (1 - z^2)^\frac{m}{2} \cdot \der{}{x}{m} P_n (z)= P_n^{(m)} (z)
	\label{equ:connLejandra}
\end{equation}
\eqref{equ:connLejandra} -- присоединённые полиномы Лежандра


\subsubsection{Ортогональность}\label{que:30}
Рассмотрим норму присоединённых полиномов Лежандра
\[
	N_{n, k}^{(m)} = \int\limits_{-1}^1 P_n^{(m)}(z)P_k^{(m)}(z)\, dz  =
	\begin{cases}
	    0 & n \neq k\\
		\neq 0 &n=k
	\end{cases}
\]

\[
    N_{n, k}^{(m)} = \int\limits_{-1}^1 (1 -z^2) \der{}{z}{m} (P_n) (1 - z^2)^\frac{m}{2} \der{}{z}{m} (P_k) \, dz = \int\limits_{-1}^1 \underbrace{(1 - z^2)^m \der{}{z}{m} P_n(z)}_\text{u} \cdot \underbrace{\der{}{z}{m} P_k(z)}_\text{dv} \, dz
\]
Интегрируем по частям
\begin{align*}
    &v = \der{}{z}{m - 1} P_k(z)\\
    &du = \der{}{z}{} \left[(1 - z^2)^m \der{}{z}{m} P_n(z) \right]
\end{align*}

%\begin{equation}
%    (1 - z^2) \derp{y}{z}{m+2} - 2 (m + 1) z \der{y}{z}{m+1} + \left[n(n + 1) - m(m + 1) \right] \der{y}{z}{m} = 0
%	\label{equ:ort1}
%\end{equation}

%\begin{equation}
%    P_n^{(m)} = (1 - z^2)^\frac{m}{2} \der{}{z}{m} P_n(z)
%	\label{equ:ort2}
%\end{equation}

\begin{multline*}
    \int\limits_{-1}^1 (1 - z^2)^m \der{}{z}{m} P_n(z) \der{}{z}{m}P_k(z) \,dz = \\ =\left. (1 - z^2)^m \der{}{z}{m} P_n (z) \der{}{z}{m - 1} P_k(z) \right|_{-1}^1 - \int\limits_{-1}^1 \der{}{z}{} \left[ (1 - z^2)^m \der{}{z}{m} P_n(z)\right] \der{P_k(z)}{z}{m - 1} \, dz
\end{multline*}
Оставшийся интеграл подставим в \eqref{equ:LejandrAtt3} и умножим на $(1 - z^2)$:
\[
	(1 - z^2)^m \der{}{y}{m + 2} - 2 (m + 1)z (1 - z^2)^m \der{}{y}{m + 1} + (1 - z^2)^m [n (n + 1) - m (m + 1)] \derp{y}{z}{m} = 0
\]

\begin{equation}
     \der{}{z}{} \left\{ (1 - z^2)^m \der{}{y}{m + 1} \right\} = - (1 - z^2)^m \left[ n(n + 1) - m(m +1) \right] \der{y}{z}{m}
	 \label{equ:ort3}
\end{equation}
Перепишем формулу \eqref{equ:ort3} $m = m' + 1 \Rightarrow m' = m - 1$

\begin{equation*}
     \der{}{z}{} \left\{ (1 - z^2)^m \der{}{y}{m} \right\} = - (1 - z^2)^{m-1} \left[ n(n + 1) - m(m - 1) \right] \der{y}{z}{m - 1}
\end{equation*}

\[
    n^2 + n - m^2 + m = (n + m) + (n^2 - m^2) = (n + m)(n - m + 1)
\]

\begin{multline*}
    N_{n,k}^{(m)} = \int\limits_{-1}^1 (1 - z^2)^{m} \der{}{z}{m} P_n(z ) \der{}{z}{m} P_k (z) \, dz =\\= \int\limits_{-1}^1 (1 - z^2)^{m - 1} \left[(n + m) (n - m + 1) \der{P_n(z)}{z}{m - 1}\right] \der{P_k(z)}{z}{m - 1}\, dz = \\ = [(n + m (n - m + 1))] \int\limits_{-1}
^1 (1 - z^2)^{m - 1} \der{}{z}{m - 1} P_n(z) \der{}{z}{m - 1} P_k(z) \, dz = \\ = [(n + m)(n - m + 1)]N_{n,k}^{(m - 1)} = (n + m) (n - m + 1) (n - m +2) N_{n, k}^{m - 2} = \\ = (n + m)(n + m - 1)\ldots (n + 1) (n - m + 1) (n - m + 2)\ldots n N_{n,k}^{(0)}
\end{multline*}
Вспомним определение факториала
\[
	n! = 1 \cdot 2 \cdot 3 \ldots n
\]
\[
   \frac{n(n - 1)(n + 2) \ldots (n - m + 2) (n - m + 1) (n - m) (n - m + 1)}{(n - m)(n - m - 1)} = \frac{n!}{(n - m)!}
\]

\[
    \frac{(n + m) ( \ldots) (n + 1) \ldots (n) (n - 1) \dots 1}{n (n - 1) \ldots 1} = \frac{(n + m)!}{n!} = \frac{n!}{(n - m)!}
\]
\[
	N_{n,k}^{(m)} = \frac{(n + m)! n!}{(n - m)!n!} N_{n, k}^{(0)}
\]
Окончательно получаем:
\begin{equation}
    N_{n,k}^{(m)} = \int\limits_{-1}^1 (1 - z^2)^m \der{}{z}{m} P_n(z) \der{}{z}{m} P_k (z) \, dz = \frac{(n + m)! }{(n - m)! } N_{n,k}^{(0)}
\end{equation}
Используя свойство ортогональности
\[
    \int\limits_{-1}^1 P_n^{(m)} (z) P_k^{(m)} \, dz = 
	\begin{cases}
	    0 &k \neq m\\
	\frac{2}{2n + 1} &k=m
	\end{cases}
\]
последнее равенство может быть представлено так:
\begin{equation}
    N_{n,k}^{(m)} = \frac{(n + m)! }{(n - m)! } \frac{2}{2n + 1}, \qquad \text{если}\quad m = k
\end{equation}
