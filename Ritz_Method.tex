\setcounter{equation}{0}
Метод Ритца служит для приближенного решения вариационной задачи. \href{http://reslib.com/book/Chislennie_metodi_analiza__Priblizhenie_funkcij__differencialjnie_i_integraljnie_uravneniya/321}{Описание.} %Зададимся несколькими функциями $f_1(x) , f_2(x) ,\ldots , f_n(x)$, каждая из которых удовлетворяет геометрическим граничным условиям задачи, и образум функцию $f(x)$ как сумму
%\begin{equation}
%	f(x) = C_1 f_1(x) +C_2 f_2(x)+...+C_n f_n(x).
%	\label{equ:equRitz1}
%\end{equation}
%Если эту функцию подставить в формулу Рэлея
%\begin{equation}
%	w^2 = \frac{\int\limits_0^l E J(f'')^2\, dx}{\int\limits_0^l m f^2 \, dx},
%	\label{equ:equRitzReley}
%\end{equation}
%то результат будет зависеть от конкретного выбора коэффициентов $C_1 , C_2 , …, C_n$.

%Метод Ритца основан на простой идее: коэффициенты $C_1 , C_2 , …, C_n$ должны быть выбраны так, чтобы вычисление по \eqref{equ:equRitzReley} дало наименьшее значение для $w^2$. Из теоремы Рэлея вытекает, что такой выбор будет наилучшим (при данной системе функций $f_i$).

%Условия минимума $w^2$ имеют вид
%\[
%	\derp{}{C_i}{}  \frac{\int\limits_0^l E J(f'')^2\, dx}{\int\limits_0^l m f^2 \, dx} = 0,
%\]
%где $i = 1, 2, \ldots, n$.
%т.е.
%\[
%	\left[\derp{}{C_i}{} \int\limits_0^l E J(f'')^2\, dx\right] \left[\int\limits_0^l m f^2 \, dx \right] - \left[[\derp{}{C_i}{} \int\limits_0^l m f^2\, dx \right] \left[\int\limits_0^l E J(f'')^2 \, dx \right] = 0
%\]

%Разделив это уравнение на интеграл $\int\limits_0^l m f^2 \, dx$ и учитывая \eqref{equ:equRitzReley}, получим

%\begin{equation}
%	\derp{}{C_i}{} \int\limits_0^l \left[ EJ(f'')^2 - w^2 m f^2\right]\, dx = 0
%	\label{equ:equRitzReley2}
%\end{equation}

%Уравнения \eqref{equ:equRitzReley2} однородны и линейны относительно $C_1 , C_2 , …, C_n$ и их число равно числу членов выражения \eqref{equ:equRitz1}. Приравнивая нулю определитель, составленный из коэффициентов при $C_1 , C_2 , …, C_n$, получим частотное уравнение. Это уравнение не только дает хорошее приближение для низшей частоты, но также определяет (хотя и с меньшей точностью) значения высших частот; при этом можно будет вычислить столько частот, сколько слагаемых принято в выражении \eqref{equ:equRitz1}.

%Метод Ритца, как и метод Рэлея, позволяет решить задачу в случаях разрывных функций $EJ$ и $m$ и когда эти функции представлены различными аналитическими выражениями на различных участках.\\

%Иногда та же идея используется в иной форме. Например, при исследовании поперечных колебаний турбинных лопаток задаются функцией $f(x) = ax^s$ (начало координат в закрепленном конце). Применяя затем формулу Рэлея \eqref{equ:equRitzReley}, получают частоту в виде зависимости от показателя степени $s$. Затем при помощи числовых расчетов определяют значение $s$, которому отвечает наименьшая частота. Это позволяет достаточно надежно определить как форму, так и частоту колебаний первого тона.\\

\begin{example}{Найти решение с помощью метода Ритца}
Подобрать полную систему линейно независимых функций $\varphi_1(x, y), \varphi_2(x, y), \ldots \varphi_n(x, y)$, удовлетворяющих граничному условию $\varphi_i|_\gamma = 0$ (все функции на граниче должны быть равны $0$).

Чаще всего это тригонометрические или степенные функции. В нашем случае удобнее выбрать степенные: 
\begin{align*}
	(1 - x^2)(1 - y^2) = \varphi_1\\
	(1 - x^2)(1 - y^2)x = \varphi_2
\end{align*}
Искомую функцию $u$ мы будем искать в виде ряда 
\[
	\sum\limits_{k = 1}^n C_k \varphi_k (x, y)
\]
где $C_k$ --- неизвестные константы.

\begin{multline*}
	\iint\limits_D \biggl\{ \left(\sum\limits_{k = 1}^n C_k \derp{\varphi_k}{x}{} \right)^2 + \left(\sum\limits_{k = 1}^n C_k \derp{\varphi_k}{y}{} \right)^2 - 2 \sum\limits_{k = 1}^n C_k \varphi_x(x, y) \biggl\}\, dx dy = \\
	= \iint\limits_D \left\{ \sum_{k = 1}^n \sum_{k = 1}^n C_k C_m \derp{\varphi_k}{x}{} \derp{\varphi_m}{x}{} + \sum_{k = 1}^n \sum_{k = 1}^n C_k C_m \derp{\varphi_k}{y}{} \derp{\varphi_m}{y}{} - 2 \sum\limits_{k = 1}^n C_k \varphi_k (x, y) \right\}\, dx dy
\end{multline*}
Так как сумма конечная, меняем местами знаки суммы и интеграла:
\[
	\sum_{k = 1}^n \sum_{k = 1}^n C_k C_m \left[ \iint\limits_D \underset{a_{km}}{\left(\derp{\varphi_k}{x}{} \derp{\varphi_m}{x}{} + \derp{\varphi_k}{y}{} \derp{\varphi_m}{y}{} \right)}\, dx dy \right] - 2 \sum\limits_{k = 1}^n C_k \iint\limits_D \underset{b_k}{\varphi_k (x, y)}\, dx dy
\]

В итоге получили
\[
	I[u] =  \sum_{k = 1}^n \sum_{k = 1}^n C_k C_m a_{km} - 2 \sum\limits_{k = 1}^n C_k b_k
\]
Константы $C_1, C_2, \ldots, C_n$ --- должны быть подобраны так, чтобы функционал $I [C_1, C_2, \ldots, C_n]$ принимал минимальные значения.
\[
	\derp{I}{C_1}{} = 0 \quad \derp{I}{C_2}{} = 0 \quad \derp{I}{C_n}{} = 0
\]
В результате получаем СЛАУ:

\[
	\begin{matrix}
	   a_{11} C_1 + a_{12} C_2 +\ldots +a_{1n}C_n = 2 b_1\\
	   \hdotsfor{1}\\
	   a_{n1} C_1 +a_{n2} C_2+ \ldots+ a_{nn}C_n = 2 b_n
	\end{matrix}
\]
Пусть решение состоит из 1-го слагаемого:
\[
	u(x, y) = C_1 (1 - x^2)(1 - y^2)
\]
Требуется найти $C_1$:

\begin{align*}
	&\iint\limits_D \left\{ \left[C_1 (1 - y^2)(- 2 x) \right]^2 + \left[C_1 (1 - x^2)(-2y) \right]^2 - 2 C_1 (1 - x^2)(1 - y^2) \right\}\, dx dy =\\
	&=4 C_1^2  \int\limits_{-1}^1 \int\limits_{-1}^1 \left[ x^2 (1 - y^2)^2 + y^2 (1- x^2)^2 \right]\, dx dy  - 2 C_1 \int\limits_{-1}^1 \int\limits_{-1}^1 (1 - x^2)(1 - y^2)\, dx dy =\\
	&=8 C_1^2 \int\limits_{-1}^1 \left[x^2 \left(1 - \frac{2}{3} + \frac{1}{5} \right) + \frac{1}{5} \left(1 - 2 x^2 + x^4 \right)\right]\, dx - 4 C_1 \int\limits_{-1}^1  (1 - x^2) \left(1 - \frac{1}{3}\right) \, dx =\\
	&=16 C_1^2 \left\{ \left(1 - \frac{2}{3} + \frac{1}{5} \right) \frac{x^3}{3} + \frac{1}{3} \left(x - \frac{2}{3} x^3 + \frac{1}{5} x^5 \right)\right\} \Bigg|_0^1 - 8 C_1 \left(x - \frac{x^3}{3} \right) \left(1 - \frac{1}{3}\right)\Bigg|_0^1 =\\
	&=16 C_1^2 \left\{\frac{1}{3} \frac{8}{15} + \frac{1}{3} \frac{8}{15} \right\} - 8 C_1 \left(\frac{4}{9} \right)\\
\end{align*}

\[
	\frac{16 \cdot 16}{3 \cdot 15} C_1^2 - \frac{32}{9} C_1
\]

\[
	C_1 = \frac{5}{16}
\]

\[
	u(x,y) = \frac{5}{16} (1 - x^2)(1 - y^2)
\]
Точность $1.5\%$.
\end{example}
