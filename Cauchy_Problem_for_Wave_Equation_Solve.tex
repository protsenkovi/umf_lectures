\setcounter{equation}{0}
\begin{equation}
	\derp{u}{t}{2} = a^2 \derp{u}{x}{2} 
	\label{equ:equInfStringCauchy}
\end{equation}
Начальные условия 
\begin{alignat*}{1}
    u(x, 0) &= f(x)\\
    \derp{u}{t}{} (x, 0) &= g(x)
\end{alignat*}
Преобразуем это уравнение к каноническому виду, содержащему смешанную производную. Уравнение характеристик
\[
	dx^2 - a^2\, dt^2 = 0
\]
распадается на два уравнения:
\[
	dx - a\, dt = 0, \quad dx + a\, dt = 0.
\]
	Характеристиками уравненения являются две прямые:
\begin{align*}
	&\xi = x - at = C_1\\
	&\eta = x + at = C_2
\end{align*}
Приводим к каноническому виду
\begin{alignat*}{2}
	\derp{u}{t}{} &= a \derp{u}{\eta}{} - a \derp{u}{\xi}{} \quad &\derp{u}{t}{2} &= a^2 \derp{u}{\eta}{2} - 2 a^2 \derps{u}{\xi}{\eta} + a^2 \derp{u}{\xi}{2}\\
	\derp{u}{x}{} &=\derp{u}{\xi}{} + \derp{u}{\eta}{} \quad &\derp{u}{x}{2} &= \derp{u}{\xi}{2} + 2 \derps{u}{\xi}{\eta} + \derp{u}{\eta}{2}
\end{alignat*}
\[
	4 a^2 \derps{u}{\xi}{\eta} = 0
\]
В итоге уравнение колебаний струны преобразуется к виду:
\[
	\derps{u}{\xi}{\eta} = 0
\]
Найдём общий интеграл последнего уравнения:
\[
	\derp{u}{\eta}{} = \chi \quad \derp{\chi}{\xi}{} = 0
\]
\[
	\chi =V(\eta) \quad \derp{u}{\eta}{} = V(\eta)
\]
\[
	u(\xi, \eta)= \int\limits_{\eta_0}^{\eta} V(\tau) d \tau + \psi(\xi)
\]
\[
	u(\xi, \eta) = \varphi(\eta) + \psi (\xi)
\]
Функция
\begin{equation}
      u(x,t) = \varphi (x +at) + \psi (x - at)
	\label{equ:equCommonIntegral}
\end{equation}	

является общим интегралом уравнения \eqref{equ:equInfStringCauchy}.

Решение задачи называется корректным если оно существует, единственно и устойчиво.\\

\textbf{Существование}\\
	$\varphi(x +at)$ и $\psi(x - at)$ - должны допускать непрерывные частные производные.\\

\textbf{Единственность}\\
	Пусть $u(x, t) = \varphi (x + at) + \psi (x - at)$. Определим $f(x)$ и $g(x)$ таким образом, чтобы удовлетворялись начальные условия:
\begin{equation}
	u(x, 0) = \varphi (x) + \psi(x) = f(x)
	\label{equ:wave1}
\end{equation}
\[
	\derp{u}{t}{} (x, 0) = a \varphi'(x) - a \psi' (x) = g(x)
\]
Интегрируя второе равенство, получим:
\begin{equation}
	a \varphi(x) - a \psi(x) = \int\limits_{x_0}^{x} g(\tau) d \tau  + C
	\label{equ:wave2}
\end{equation}
где $x_0$ и $C$ - постоянные. Из \eqref{equ:wave1} и \eqref{equ:wave2} находим:
\[
	2 \varphi(x) = f(x) + \frac{1}{a} \int\limits_{x_0}^x g(\tau) d \tau + \frac{C}{2}
\]
\[
	2 \psi(x) = f(x) - \frac{1}{a} \int\limits_{x_0}^x g(\tau) d \tau - \frac{C}{2} 
\]
\begin{equation}
\begin{cases}
	 \varphi(x)  = \frac{1}{2} f(x) + \frac{1}{2a} \int\limits_{x_0}^x g(\tau) d \tau\\
	\psi(x) = \frac{1}{2} f(x) - \frac{1}{2a} \int\limits_{x_0}^x g(\tau) d \tau
\end{cases}
\label{equ:equWave3}
\end{equation}
Таким образом, мы определили функции $\varphi$ и $\xi$ через заданные $f$ и $g$, причём равенства \eqref{equ:equWave3} должны иметь место для любого значения аргумента\footnote{В формуле \eqref{equ:equCommonIntegral} функции $\varphi$ и $\xi$ определены неоднозначно. Если от $\varphi$ отнять, а к $\xi$ прибавить некоторую постоянную $C_1$, то $u$ не изменится. В формуле \eqref{equ:equWave3} постоянная $C$ не определяется через $\varphi$ и $\xi$, однако мы можем её отбросить, не меняя значения~$u$. При сложении $\varphi$ и $\xi$ слагаемые $\frac{C}{2}$ и $-\frac{C}{2}$  уничтожаются.}.
\begin{equation}
	u(x, t) = \frac{f(x + at) + f(x - at)}{2} + \frac{1}{2a} \int\limits_{x - at}^{x + at} g (\tau) d \tau
	\label{equ:equDalamber}
\end{equation}

Формулу \eqref{equ:equDalamber}, называемую \textit{формулой Даламбера}, мы получили, предполагая существование решения поставленной задачи. Эта формула доказывает единственность решения. Если бы существовало второе решение задачи \eqref{equ:equInfStringCauchy}, то оно представлялось бы формулой \eqref{equ:equDalamber} и совпадало бы с решением.\\

\textbf{Устойчивость}\\
Рассмотрим решение возмущённой задачи $\tilde u$:
\[
	\tilde u = f(x) + \varepsilon_1 \quad \derp{\tilde u}{t}{} (x, 0) = g(x) + \varepsilon_2
\]
Рассмотрим разность решений исходной и возмущённой задач
\[
	\abs{u(x, t) - \tilde u (x, t)} \leq \abs{\varepsilon_1} + \frac{1}{2a} \int\limits_{x - at}^{x + at} \varepsilon_2\, dt \leq \sigma(1 + t) \Rightarrow
\]
$\Rightarrow$ $\forall \varepsilon >0$ можно подобрать $\sigma: \sigma \leq \frac{\varepsilon}{1 + t}$\\
Следовательно решение устойчиво.
		


