Рассмотрим задачу Коши:
\[
	\derp{u}{t}{} = a^2 \derp{u}{x}{2}, \quad t > 0\\
\]
Начальное условие:
\[
	u(x, 0) = f(x)
\]
\[
	t a^2 = T \Rightarrow \derp{u}{T}{} = \derp{u}{x}{2}
\]
Будем искать ограниченное нетривиальное решение уравнения методом разделения переменных, представимое в виде
\[
	u = X(x) Y(t) 
\]
Подставляя это выражение в исходное уравнение, получаем:
\[
	\quad \frac{X''}{X} = \frac{Y'}{Y} = - \lambda^2,
\]
где $\lambda^2$ -- параметр разделения.\\
Частные решения находятся в таком виде
\begin{align*}
	&X(x) = \alpha (\lambda) \sin \lambda x + \beta (\lambda) \cos \lambda x\\
	&Y(t) = e^{- \lambda^2 t}
\end{align*}
Общее решение
\[u(x, t) = (\alpha (\lambda) \sin \lambda x + \beta(\lambda) \cos \lambda x) e^{-\lambda^2 t}\] 
Запишем интегральный вид уравнения
\[u(x, t) = \int\limits_{-\infty}^{+ \infty} (\alpha (\lambda) \sin \lambda x + \beta (\lambda) \cos \lambda x) e^{-\lambda^2 t} \,  d \lambda\]
Из начальных условий найдём неизвестные
\[u(x, 0) = \int\limits_{-\infty}^{+ \infty} (\alpha (\lambda) \sin \lambda x + \beta (\lambda) \cos \lambda x) d\lambda = f(x)\]
\[\alpha (\lambda) = \frac{1}{2 \pi} \int\limits_{- \infty}^{+ \infty} f (\xi) \sin \lambda \xi \, d \xi\]
\[\beta(\lambda) = \frac{1}{2 \pi} \int\limits_{- \infty}^{+ \infty} f (\xi) \cos \lambda \xi \, d \xi\]
Подставляя в уравнение и меняя порядок интегрирования получим
\[u(x,t) = \frac{1}{2 \pi} \iint\limits_{R^2} f(\xi) (\sin \lambda x \sin \lambda \xi + \cos \lambda x \cos \lambda \xi) e^{- \lambda^2 t} \, d \lambda d \xi \]
\[u(x, t) = \frac{1}{2 \pi} \iint\limits_{R^2} f(\xi) \cos \lambda (x - \xi) e^{- \lambda^2 t} \, d \lambda  d \xi\]
Произведём замену переменных
\[
	\int\limits_{- \infty}^{ + \infty} \cos \lambda (x - \xi) e^{- \lambda^2 t} \, d \lambda = \left[ 
		\begin{tabular}{l}
			$\lambda \sqrt{t} = \sigma$\\ 
			$\lambda(x - \xi) = \sigma \omega$ 
		\end{tabular} \right] 
	= \int\limits_{- \infty}^{ + \infty} \cos \sigma \omega e^{- \sigma^2} \frac{d \sigma}{\sqrt{t}}
\]
Возьмём этот интаграл. Для удобства обозначим его как
\[I(\omega) = \int\limits_{- \infty}^{ + \infty} \cos \omega \sigma e^{- \sigma^2} \, d \sigma\]
Найдём производную по $\omega$
\[\der{I}{\omega}{} = - \int\limits_{- \infty}^{ + \infty}\sigma \sin \sigma \omega e^{- \sigma^2} d \sigma - \frac{\omega}{2} \int\limits_{- \infty}^{ + \infty} \cos \sigma \omega e^{-\sigma^2} d \sigma\]
\[I(\omega) = \sqrt{\pi} e^{- \frac{(x - \xi)^2}{4}} = \sqrt{\pi} e^{-\frac{\omega^2}{4}}\]
В итоге получаем
\[\int\limits_{- \infty}^{ + \infty} \cos (x - \xi) e^{- \lambda^2 t} \, d \lambda = \sqrt{\frac{\pi}{t}}  e^{- \frac{(x - \xi)^2}{4 t}}\]
Подставим в общее решение
\[u(x, t) = \sqrt{\frac{\pi}{t}} \int\limits_{- \infty}^{ + \infty} f(\xi) e^{- \frac{(x - \xi)^2}{4 t}} \, d \xi \cdot \frac{1}{2 \pi}\]
Окончательное решение:\\
\[u(x, t) = \frac{1}{2 a \sqrt{\pi t}} \int\limits_{- \infty}^{ + \infty} f(\xi)  e^{- \frac{(x - \xi)^2}{4 a^2 t}} \, d \xi\]\\
Проверим, что это действительно решение:\\
\[t=0 \quad \lim\limits_{t \downarrow 0} u(x, t) = \int\limits_{- \infty}^{ + \infty} f(\xi) \delta (\xi - x) \, d \xi = f(x)\]
\[G(x, \xi, t) = \frac{e^{- \frac{(x - \xi)^2}{4 a^2 t}}}{2 a \sqrt{\pi t}}\]
Продифференциируем функцию $G(x, \xi, t)$
\begin{align*}
	&\derp{G}{t}{} = \left( \frac{(x - \xi)^2}{2\cdot 4 a^3 t^2 \sqrt{t}} -\frac{1}{4 a t \sqrt{\pi t}}  \right) e^{- \frac{(x - \xi)^2}{4 a^2 t}}\\
	&\derp{G}{x}{} = - \frac{\cancel 2 (x - \xi)^2}{4 a^2 t} \cdot \frac{1}{\cancel 2 a \sqrt{\pi t}} e^{- \frac{(x - \xi)^2}{4 a^2 t}}\\
	&\derp{G}{x}{2} = \left(\frac{2 (x - \xi)^2}{16 a^5 \sqrt{\pi t} t^2} - \frac{1}{4 a^3 t \sqrt{\pi t}} \right) e^{- \frac{(x - \xi)^2}{4 a^2 t}}
\end{align*}
Функция удовлетворяет уравнению теплопроводности по переменным $(x, t)$.

Итак, мы пришли к интегральному представлению искомого решения
\[
	u(x, t) = \int\limits_{- \infty}^{ + \infty} f(\xi)G(\xi, x, t) \, d \xi ,
\]
где
\begin{equation}
	G(x, \xi, t) = \frac{1}{2 a \sqrt{\pi t}} e^{- \frac{(x - \xi)^2}{4 a^2 t}}
	\label{equ:koshiInftyG}
\end{equation}
			Функцию $G(x, \xi, t)$, определяемую формулой \eqref{equ:koshiInftyG}, часто называют фундаментальным решением уравнения теплопроводности.