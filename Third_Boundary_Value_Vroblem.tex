Рассмотрим задачу для пространственной области. Третья краевая задача отличается от первых двух краевыми условиями:\\
\[
	\Delta u = 0
\]
\[
	\derp{u}{n}{} + h u \Big|_S = f
\]
Функцию Грина ищем в таком виде\\
\[
	G = W - \frac{1}{r_{AP}}
\]
Потребуем,чтобы на границе \[
	\derp{G}{n}{} + h G|_S = 0
\]
Тогда  формулу Грина можно переписать\\
\begin{multline*}
	u(x_0, y_0, z_0) = \frac{1}{4 \pi} \iint\limits_S \left( u \derp{G}{n}{} - G \derp{u}{n}{} \right) dS = \frac{1}{4 \pi} \iint\limits_S \left( u \derp{G}{n}{} + u h G - u h G - G \derp{u}{n}{}\right) dS = \\ = \derp{1}{4 \pi}{} \iint\limits_S \left[ u \left( \derp{G}{n}{} + h G \right) G \left( \derp{u}{n}{} + h u\right)\right] dS = - \frac{1}{4 \pi} \iint\limits G f dS
\end{multline*}
До сих пор мы рассматривали задачу для $(x_0, y_0, z_0) \subset \Omega \backslash S, S = \partial\Omega$. 
Рассмотрим предельный случай, когда точка $a$ находится на границе области.
Будем огрничивать особую точку полусферой\\
Площадь полусферы $2 \pi R^2$.
\[
	u(x_0, y_0, z_0) = \frac{1}{2 \pi} \iint \left(u \derp{v}{n}{} - v \derp{i}{n}{} \right)\, dS - \frac{1}{2 \pi} \iiint\limits_\Omega \left(u\Delta v - v \Delta u \right)\, d\Omega
\]
Если $a \notin \Omega$, то 
\[
	\iint\limits_S \left(u \derp{v}{n}{} - v \derp{u}{n}{} \right)\, dS - \iiint\limits_\Omega \left(u\Delta v - v \Delta u \right)\, d\Omega = 0
\]
\[
	u(x_0, y_0, z_0) = \begin{cases} 
						4 \pi u & A \in \Omega \\
						2 \pi u & A \in S \\
						0 &  A  \notin  \Omega
					\end{cases}
\]


